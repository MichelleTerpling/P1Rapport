\chapter{Interview: Farstrup boldklub}\label{ch:appBlabel}
Interviewer: Liv Holm\\
Interviewperson: Lars Segerstrøm
\\
\\
0:00
\\\\
Interviewer: Den kan vidst godt høre os hvis vi bare snakker sådan her. Sådan!
\\\\
Interviewer: Hvordan er det så med jeres egen faciliteter her, hallen og sådan noget. Jeg ved jo at der er en halstyrer der ejer den, hvordan fungerer det?
\\\\
Lars: Jamen selve strukturen er at Farstrup-hallen er en selvejende institution. Det betyder så at der er en Aalborg kommune der står bagved, men bygningen, eller hvad skal man sige, hallen som sådan, har sin egen bestyrelse og sin egen, hvad skal man sige, drift. Så der får man lidt tilskud, og så får man at vide at nu sørger i for at få det bedste ud af det her. Og der kommer vi så som idrætsforening, og lejer os ind.
\\\\
Interviewer: Okay.
\\\\
Lars: Så hallen har sin egen økonomi, og så er der en boldklub der lejer sig ind, og bruger de timer vi nu kan, kan besætte. Og det er frit for alle, så der kunne opstå en forening herude i morgen, eller en forening fra Løgstør kunne i princippet komme og sige at der er pres på haltiderne i Løgstør, så vi kunne godt tænke os at leje os ind, så det er frit for alle.
\\\\
Interviewer: <Bekræftende> Det er egentlig frit for alle. Okay, godt </>.
\\\\\\
1:04
\\\\
Interviewer: Jeg ved også at der er nogle redskaber og sådan noget i hallen, er det hallen der ejer det, eller er det skolen, fordi jeg ved at de også bruger det og sådan noget, eller….? <ledende>.
\\\\
Lars: Jamen det er delt mellem idræt, hvad skal man sige, altså vores forening, og så skolen. Og jeg tror at det kan sagtens være at der er nogle redskaber tilbage fra da hallen startede i midt halvfjerdserne, der stadig fungerer, som vi nok fik som en del, for ligesom at sige at det er en startpakke.
\\\\
Interviewer: <Bekræftende> Okay, ja </>.
\\\\
Lars: Men ellers så er det når der skal en ny måtte-et-eller-andet, jamen så er det boldklubbens gymnastikafdeling der køber den ind.
\\\\
1:45
\\\\
Interviewer: <B> Okay, godt </>. Hvad med banerne udenfor, er der nogen der er herovre? <gestikulerer mod banerne>. Hvordan fungerer de? Er det også….? <ledende>
\\\\
Lars: Det er kommunens areal, og det er dem der sørger for at vedligeholde træer, græs og det hele.
\\\\
Interviewer: <B> Okay </>.
\\\\
Lars: Og så hjælper vi selvfølgelig til. Vi har lige fået bygget en hytte derovre, til vores bolde og også et par toiletter.
\\\\
Interviewer: Ja, det har jeg set, at der har været noget i gang derovre.
\\\\
Lars: <B> Ja </>. Men det er også kommunen. Altså, vi får pengene og så bygger vi, og så giver vi nøglerne til kommunen, og så bruger vi det så bagefter, mod en eller anden leje.
\\\\
2:25
\\\\
Interviewer: Har i nogen styr på hvordan, altså hvor meget jeres faciliteter bliver brugt, og alle de ting i ejer, hvornår de bliver slidt og sådan noget?
\\\\
Lars: Altså når vi har sådan noget som tøj, det er noget af det der slides mest på.
\\\\
Interviewer: <B> Ja, selvfølgelig </>
\\\\
Lars: Sådan en gang håndbold og lidt harpiks, det kan godt gøre det…
\\\\
Interviewer: Ja, ret beskidt.
\\\\
<Ler sammen>
\\\\
Lars: Ja, så det er der hvor vi har den største omsætning, så er det klart at håndbolde og fodbolde, det er også der vi har en stor omsætning.
\\\\
Lars: På trøjer, der har vi en trøje-ansvarlig i bestyrelsen, der sørger for at man nu køber de rigtige sæt ind. Vi har et anderkendt mærke, og det er måske lidt dyrere, men så er det lettere at supplere hvis en trøje går i stykker.
\\\\
Interviewer: Fordi det bliver det samme sådan…? <ledende>
\\\\
Lars: Så det bliver det samme, ja, og hvis det så lige er nummer seks fordi det er stregspilleren, der er fyldt med harpiks og ikke kan rigtig gøres rent igen, jamen så har vi mulighed for at vi næste år kan købe en ny nummer seks. Så kan det godt ske at Hummel-logoet det ser lidt anderledes ud, men de har sådan en standardtrøje der sådan ikke forandrer sig ret meget.
\\\\
Interviewer: <B> Okay, det lyder godt </>. Så der er simpelthen en der har ansvaret for det?
\\\\
Lars: <B> Ja </>.
\\\\
Interviewer: Hvad med sådan, andre poster I har? I er vel nogle flere mennesker i bestyrelsen?
\\\\
Lars: Ja, vi sidder fem i bestyrelsen.
\\\\
Interviewer: Fem i bestyrelsen?
\\\\
Lars: Ja, og så har vi de her fire sportsgrene. Vi har håndbold, fodbold, gymnastik og badminton, hvor de to store det er fodbold og håndbold.
\\\\
Interviewer: <B> Ja, selvfølgelig </>
\\\\
Lars: Og der sidder to af bestyrelsesmedlemmerne som har ansvaret for det, sammen med et udvalg. Så der sidder 2 bestyrelsesmedlemmer i udvalget, sammen med nogle frivillige. Det kunne være trænere eller andre der har interesse for det.
\\\\
Interviewer: <B> Ja </>.
\\\\
Lars: Så de styrer om der skal købes nye fodbolde ind eller nu-her til vintersæsonen hvor er det vi lige opbevarer dem så de ikke knækker(/flækker) og frosner.  
\\\\
4:22
\\\\
Interviewer: </B>Jaer, selvfølgelig, selvfølgelig </>. Hvad med sådan noget som, årh hvad er det nu jeg tænker på. Det kan jeg ikke lige komme i tanke om nu. Jo, hvordan er det i får styr på at lave den organisering. Altså, er det noget man kan blive sådan valgt ind til i bestyrelsen eller…? <ledende>
\\\\
Lars: Ja, i bestyrelsesdelen, der er man valgt - i forbindelse med generalforsamlinger. Så der sidder man for to år af gangen, så halvdelen, nu er vi så fem, men to det ene år og tre det andet år, der skal enten genvælges eller knyttes/kendes nye til. Og udvalgene, der bliver man udpeget, der bliver man spurgt om man kunne tænke sig og give et nap med, og der sidder man ikke sådan for en fast periode.
\\\\
Interviewer: <B> Nej, nej, der siger man bare hvornår man er… </><ledende>
\\\\
Lars: <B> Ja </>. Så når der starter en fodboldsæson, jamen så finder man nye medlemmer til udvalget, og nogle bliver gengangere og andre, jamen det er måske nye ansigter i klubben.
\\\\
5:20
\\\\
Interviewer: Er der nogen der får løn for og være her eller….?
\\\\
Lars: Nej.
\\\\
Interviewer: <På samme tid> Nej. </>. Det er alt sammen frivilligt arbejde.
\\\\
Lars: <B> Det er frivilligt arbejde </>. Man får dækket sin omkostninger.
\\\\
Interviewer: <B> Ja </>.
\\\\
Lars: Så for en træner så, man kan vælge at gøre det på den måde, hvor man siger at jeg har kørt så og så mange kilometer, eller jeg har købt frimærker, eller hvad pokker det er, så får man det efter de kvitteringer der er (?). Der er en anden mulighed, som vi har valgt, det er at sige at jamen vi betaler tusind kroner til en træner, og så hvis det er der sker et eller andet hvor træneren siger mine omkostninger har simpelthen oversteget de tusind kroner, jamen så kigger vi på det. Men vi kigger ikke bilag igennem, det bliver alt for administrativt.
\\\\
Interviewer: <B> Okay, okay, det kan jeg også godt se </>.
\\\\
6:07
\\\\
Interviewer: Men hvordan er det så at i får skaffet sådan, nok frivillige ind? Jeg ved jo at det ofte er det der med at nogle ældre teenagere der tager sig af de mindre børn, og træner dem og sådan noget og…?
\\\\
Lars: Jamen, det er en evig udfordring og få frivillige.
\\\\
Interviewer: Det er det?
\\\\
Lars: Ja. Specielt til bestyrelsesdelen. Det er noget lettere til der hvor du er tættere på gulvet, du er tættere dine medlemmer. Fordi der kan forældrene eksempelvis…
\\\\
Interviewer: Bare træde ind…
\\\\
Lars: Ja, så træder de ind, fordi den time de bruger på gymnastiktimen eller på gymnastikarrangementet, jamen der er de jo sammen med dem de kan lide (medlemmerne). Det at være en del af bestyrelsen, jamen der skal du jo sådan sige farvel til dem om aftenen, og så sætter du dig til et bestyrelsesmøde, og det er knap så synligt.
\\\\
Interviewer: <B> Ja, selvfølgelig </>.
\\\\
Lars: Så det andet (træner) det er også lettere og forholde sig til fordi man ved, hvis ikke man har prøvet det får, så kan man se jamen hvordan er det en træner agerer. Og hvad laver en bestyrelse? Det er lidt mere svært, mere udefinerbart.
\\\\
Interviewer: <B> Ja, ja. Det er også derfor vi er her i dag, for at høre om hvad det egentlig er der sker her </>
\\\\
<ler sammen>
\\\\
7:13
\\\\
Interviewer: Men hvordan organiserer i så jeres frivillige, fungerer det hele fra mund til mund aftaler, om hvornår tingene skal ske og sådan noget, eller…?
\\\\
Lars: Ja, man kan sige, typisk når vi afslutter en håndboldsæson eller en fodboldsæson, så spørger vi jo til hvem er klar på holdet næste år, sådan så vi lige i så god tid som muligt har en fornemmelse af, både hvor mange spillere vi har, men også hvor mange trænere vi har. Og typisk, så er der jo gengangere der siger jeg er klar igen om et halvt års tid. Så har man også haft sin pause og blevet sulten igen, både som træner, men også som spiller.
\\\\
Interviewer: <B> Ja, selvfølgelig </>
\\\\
7:51
\\\\
Lars: Så det er måden, sådan vi starter, og så er det klart så der hvor der er huller. Der går vi så i gang med at spørge til, at måske  trække på nogen der har været træner før, kigger på den overgang[årgang?], der nu mangler en træner til, og siger er nogle oplagte forældre, at have fat i?
\\\\
Interviewer: have fat i og have spurgt, ja
\\\\
Lars: Vi har valgt for nogle år siden, at køre med en meget, meget lav kontingentsats fordi vi lever i et sådant udkantsområde hvor at når vi kan finde otte drenge i samme årgang, jamen så skal vi gerne have til at spille fodbold alle otte, eller spille håndbold alle otte, fordi at så kan vi stille hold. Og derfor så må økonomien ikke være et issue.
Så lige i øjeblikket er vi i gang med en test, hvor vi laver et total kontingent. Så en på 10 år giver 600 kr, og så kan man være med i alle fire sportsgrene for 600 kr. Håndbold, foldbold, badminton og gymnastik. 
\\\\
Interviewer: nå okay ja, det lyder som et godt tilbud.
\\\\
Lars:  Så det er jo et rigtig godt tilbud ja. Det er også for at sige at det koster ikke 300 kr. Per gang. Eller per sports aktivitet. Fordi [hvis?] man så har været med til to ting, så kan det godt være at forældrerne siger nej til den tredje ting, fordi så skal de have penge op af lommen igen, det skal de ikke her. 
\\\\
Interviewer:  Det er sådan at alle er en del af det hele, sådan på en måde. 
\\\\
Lars: Så der håber vi jo på at kan få flere til at deltage mere 
\\\\
9:12 Interviewer: Ja selvfølgelig.Hvad med sådan noget som at, nu er det jo håndbold, og så kommer badminton ind over også , og tager noget af hallen. Er det, de tider man sådan spiller, og de tider folk er her og træner og sådan noget. Er det bare sådan noget der kommer fra, det gjorde vi også sidste år? 
\\\\
Lars: Der er lidt ændringer undervejs. Der kan være, nu er Tina, halbestyren, siger jamen for mig ville det være rart hvis at seniorhåndbold de ligger på de og de tider, og måske samlet fordi at de har harpiks med på banen, så der skal hun jo gøre ekstra godt rent bagefter. (Interviewer: Ja selvfølgelig)\\
Og badminton, der er jo også en Sebber badmintonklub, de spiller om mandagen, og hvad kan man sige boldklubbens badminton de spiller om tirsdagen, så det hænger også sådan sammen, så er hun jo fri for at skal tage net op og ned hele tiden.(Interviewer: Ja selvfølgelig, ja)\\
Så der er nogle ønsker den vej, og så har håndbold, dem der er, hvis vi har nogle hold der sådan hvad kan man sige, udvikler sig positivt og de kunne godt tænke sig at måske træne to gange om ugen, jamen så kommer de måske med nogle ønsker,  og siger, jamen kan vi ikke gøre noget tirsdag-torsdag
\\\\
10:15
\\\\
Interviewer: okay det lyder helt fint. Jo nu tænkte jeg også på hvordan, sådan noget som turneringer. Altså nu ved jeg allermest om håndbold, fordi det er det jeg allermest selv har spillet, så jeg ved jo at man spiller som regelt, det gjorde vi i hvert fald, kampe om søndagen. Og, men jeg ved ikke om det er noget i overhovedet er inden over, sådan turneringer og sådan noget. Og hvordan man for det til at hænge sammen. 
\\\\
Lars:  Det er håndboldudvalget, Tina sidder også med i håndboldudvalget, og det gør hun, hun har jo også en stor interesse for det. Men også for at sikre at, at vi så har vores kan man sige vores stævnedage. Hvor vi måske har mange hold i gang, og at det ikke er et enkelt hold en søndag formiddag, og så kommer der igen i næste uge. At vi ligesom får måske fem hold, eller fem kampe afviklet på en søndag. Sådan så at der virkelig kommer folk i hallen, og der er gang i den og der er opbakning. Og de små spillere de ser de voksne spille og sådan noget, det giver en god klubånd.
\\\\\\\\
11:17
\\\\
Interviewer: Jamen det gør det, det kan jeg godt sige. Okay så det er simpelthen noget man godt sådan kan arrangerer med (Lars: Ja) sådan, okay.
\\\\
Lars: De enkelte forbund er meget velvillige i at kan flytte kampe. 
\\\\
Interviewer:  okay, så man kan få sådan noget til at lade sig gøre, ja.
\\\\
Lars: Så man har, når først man har fået grupperet de enkelte puljer, så bliver der lavet sådan et udkast til plan. Kan man sige kamp-plan. Og så kan man begynde og sorterer rundt og flytte på. Det gør håndboldudvalget, herunder Tina.
\\\\
11:45
\\\\
Interviewer: Jeg gætter også på at de gør lidt det samme til fodbold (Lars: ja), med er det mandag aften? Det kan jeg huske at jeg har spillet før også. 
\\\\
Lars: ja. Så der, ja det gør man også, [Nu til?] fodbold er vi, jo der er vi stærk på ungdommen, vi har ikke ret meget, der har vi ikke rigtig seniorhold. Vi har et [old?]boys hold. Og de unge kan godt lide at se voksne spille, fordi det er jo dem de ser op til, så det her med at have nogen kampe der bliver spillet sammen med de voksne, i tilknytning til. Det gør jo også at de bliver på banen lige det der lidt længere, i hallen lidt længere, lige ser og syntes de er en del af noget større.
\\\\
Interviewer: Det bliver vel også sådan en familie tur? 
\\\\
Lars: ja, så går der en hel dag
\\\\
12:30
\\\\
Interviewer: ja, det rigtig og så tænkte jeg lige, hvad kan jeg spørge om mere, hvor kom jeg fra? Jo har i nogensinde sådan problemer med at få tid nok i hallerne og eller sådan presset på det?
\\\\
Lars: Nej, da jeg var lille, ung knægt. Kan jeg huske at der var der haltider, helt op til klokken 11. Det er der ikke i dag. (Interviewer: Nej det er der ikke) fordi det efterspørges ikke, det, der er ikke rigtig nogen der ønsker at bruge det. Så jo der er kamp om de gode tider. Ja fra klokken 6 og så til klokken 9. De tre timer der er der heftig efterspørgsel. Men vi er jo begunstiget af at være den eneste forening, nu har vi Sepper badmintonklub, men ellers så er vi nogle af de eneste der bruger hallen. Vi bestemmer i høj grad selv, og vi kunne jo også. I nogen større byer, jamen så har man ikke én moderforening, der ligesom både har badminton og fodbold. Der har man foreninger for de enkelte sportsgrene. Så det har vi jo også sådan begunstiget af at, jamen vi skal blive enig med os selv, og fordele det. Så derfor så når fodbolden stopper jamen så, de der overlap der er mellem sæsonerne, med fodbold om sommeren, nu har de jo spillet fodboldkampe her indtil uge 42. Og der er selvfølgelig også håndboldtræning er jo startet, fordi den er også gået i gang. Så der er jo nogen, de her overlap der har vi en større mulighed for at koordinerer.
\\\\
14:07\\\\
Interviewer: Hvad med sådan noget som, fordi jeg tænker tit at det der med fodboldkampe nå det bliver mørkere, nu ved jeg godt de bliver også holdt nogle gange lidt senere om aftenen, er det nogensinde et problem for jer at strukturerer uden om?
\\\\
Lars: Nej, vi har lys på banen. Så det er ikke noget problem.
\\\\
14:23\\\\
Interviewer: Jo jeg tænkte faktisk på, når man er i turneringer med for eksempel fodbold eller håndbold, fungerer det ikke på den måde at børneholdene de skal spille to kampe mod det samme hold? (Lars: Jo)Og så er det enten på hjemmebane og så et på udebane med de samme. 
\\\\
Lars: De helt små hold, altså de har det er jo sådan næsten en turnering hver gang. De er ikke afsted hver uge, der er måske hver tredje uge, og så spiller de tre eller fire kampe i stedet for. Så det bliver sådan et stævne, og i DBU der tæller man ikke point for de helt små. Så der er man med og de har tre-mands kampe og de har fem-mands kampe, der går det mere sådan op i leg og de bliver guidet. Der for de lov til at kaste indkaste lige så mange gange de nu har behov for, så de lærer det.
\\\\
15:18
\\\\
Interviewer: Og så tænkte jeg på om i nogensinde bruger nogen sådan, planlægningsværktøjer [egentlig?] i klubben for at holde styr på det hele, eller bruger i papir og sådan noget?
\\\\
Lars: Jamen vi har jo, nogen har jo en stor kærlighed til noget regneark. Så det bruger vi, som styreværktøj. Til haltiderne når vi flytter rundt, og sådan ligesom for at kunne visualiserer brugen af harpiks, hvornår er det, det skal være. Fordi, der er jo også nogen seniorhold, for et par år siden havde vi to-tre seniorhold, der spillede de om onsdagen, og også til sent. Og der blev Tina lidt presset på at hun skulle nå at vaske gulvet, til næste dag når skolebørnene kommer til idræt.
\\\\
Lars: Vi styrer vores ting med gule sedler og så lidt regneark
\\\\
Interviewer: Okay, fordi vi tænkte på hvad vi kunne være en hjælp til, ikke at vi kan lave et program til jer, vi er slet ikke nået så langt endnu til at gøre sådan noget, men vi har lært en del om hvordan man programmerer ting, og hvordan man bruger en del matematik gennem programmering så man kan lave sådan nogle... en nemmere måde på bare at sige tag de her tal og så giver du mig noget tilbage. Så det er lidt også der hvor vi skulle finde et problem til, så man kan jo sagtens have noget om at f.eks. de trøjerne med harpiks eller det her harpiks situation. Det havde jeg egentlig ikke tænkt over at det var et stort problem, det kan jeg jo se at der er, hvis nu at man skal gøre rent hver gang. Også med de frivillige i bestyrelsen, når det er svært at få fat i folk og sådan noget.
\\\\
Lars: Vi forsøger, udover det her forsøg med at lave total kontingent, så har vi også forsøgt at gøre så man får en rabat hvis man byder ind med 5 timers frivilligt arbejde. Det er for at vi har nogle frivillige hænder til at bage en kage i forbindelse med et hjemmestævne eller sidde og tage imod betaling når vi har gymnastik opvisninger eller hvad det nu kunne være. Det håber vi da på, at det kan være med til også og gøre noget og det bliver jo en stor koordineringsopgave for hvem har gjort hvad, men også for at få delt de opgaver. Der er nogle gange så fristes man til i stedet for at lære en op og bruge 20 minutter på det, så kan man gøre det selv på et kvarter. Så gør man det bare på et kvarter. Det bliver til mange små opgaver så bliver man selv træt. Selvom vi er 5 i bestyrelsen så er der også mange aktiviteter vi skal have styr på, så vi har rigtig brug for hjælp for dem i udvalget, men også for kontinuitet på de her lidt vigtige ting. Det er rart nok derfor har vi de her små ikke irriterende opgaver, men små opgaver der ligesom kunne man hægte dem på rundt omkring til nogle frivillige, det kunne også være med til at få dem lidt tættere på. Så det er sådan vores naive håb.
\\\\
18:44
\\\\
Interviewer: Er der noget du tænker der lige kunne være relevant for os at vide om klubben/foreningen?  
Lars: Nej, nu har jeg prøvet at beskrive sådan de her nye tiltag. Det er så noget der er oprigtigt varmt for os lige i øjeblikket, som vi går meget op i og som vi satser på at det skal lykkedes.
\\\\
Interviewer: Så det er simpelthen lige begyndt alt det her med at få denne her rabat og … 
\\\\
Lars: Der er en udfordring med hvordan vi modtager medlemstilskud, og det gør vi efter den aktivitet den enkelte deltager i. Men for at få det tilskud så skal vi dokumentere at den enkelte får eller betaler for den enkelte sportsgren. Så her er vi udfordret i at nå men når man så betaler en gang så skal vi jo sætte kryds og vedtage at han er et medlem og deltager i gymnastik og håndbold, så er det de to sportsgrene vi kan få medlemstilskud til på den enkelte
\\\\
Interviewer: Men de kan teknisk set også bare springe på det andet, uden egentlig at…
\\\\
Lars: Ja, der er en sådan en administrativ tung opgave, det vidste vi godt inden vi gik i gang at der selvfølgelig skulle vi kunne dokumentere hvad man deltog i, og det gjorde vi også før men der var noget lettere at sige men Liv har betalt for håndbold, fint det er så det, hun deltager. Så har hun også betalt for badminton, og det deltager hun også i, der var tingene noget mere reelt.
\\\\
Interviewer: Jeg kan også huske at lige når vi startede så fik man nogle gange prøvegange til at sige okay er det noget for en, og så blev man skrevet på en seddel at nu deltog man i det. Gør man det stadig på den måde?
\\\\
Lars: Vi har fået lavet for nogle år siden en hjemmeside hvor der er sådan en betalings indgang. Så det er den vej vi kan styre det igennem.
\\\\
Interviewer: Så det er ikke længere sådan en håndskrevet…
\\\\
Lars: Nej, det tog alt for lang tid
\\\\
Interviewer: Ja, det kan jeg næsten forestille mig. Det er lidt sjovt at høre om det hele sådan at det har også ændret sig alligevel, man er ikke helt så lille længere.
\\\\
Lars: Nej, tingene er… der er også mange tilbud til foreninger, styre værktøjer, hjemmesider og kommunikations ting og sager.
\\\\
Interviewer: Vi har fundet nogle, var det fra DPU, hvor det var sådan en masse forskellige værktøjer til hvordan man kunne planlægge tingene, og det var meget sådan, det vi fandt var meget, det der med at alle sidder sammen i et rum, så sidder man med papir og blyant og så skriver man ned de forskellige forslag man har til hvordan man vil styre det, og hvordan man skal finde en leder osv.. Jeg ved ikke om det bliver for besværligt at lave noget it system på sådan noget eller...       
\\\\
Lars: Nej, der findes jo forskellige knubs ud af noget... Noget der i virkeligheden favner det hele, men der er nogen der har udviklet kommunikationsværktøjer for bestyrelsen også så du får integreret noget Facebook, sider...
\\\\
Interviewer: Ja, sider så man kan sidde og skrive sammen.
\\\\
Lars: Ja
\\\\
Interviewer: Du ved ikke hvad de værktøjer hedder?
\\\\
Lars: Jeg tror at på DBU har de et samarbejde med Conventus, men det er noget økonomisystem og noget medlemsliste- og håndtering. Vi bruger noget der hedder Klubmodul, det har både betalingssiden og medlemsregistrering og så også en hjemmeside. Og der ligger også noget kommunikation og noget organisationsdiagram som, hvis man så har nørdet i den så kan man bruge det
\\\\
Interviewer: Det gør I ikke så meget af i bestyrelsen?
\\\\
Lars: Nej, fordi det er sådan lidt det samme og hvis det andet fungerer så er det sådan. Så skriver man en SMS eller mail til de andre, også så er det måden at gøre det på.
\\\\



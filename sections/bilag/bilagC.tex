\chapter{Interview: Aalborg Flyers}\label{ch:appClabel}
Interviewer 1: Barbara Herdalur Poulsen\\
Interviewer 2: Morten Kjeldsen.\\
Interviewperson: Jonas Nielsen
\\
\\
18:43 (Start turneringer, også lidt om kidz)\\
20:40 (Forskellige turneringer. Landsturneringer)\\
21:25 (En-dags turneringer)\\
23:43 (Planlægning af turnering)\\
27:38 (Framelding)\\
30:00 (Tid brugt på planlægning)
\\
\\
18:43
\\\\
Barb: Så har jeg sådan en liste, hvor vi har hvad for nogle logistiske opgaver der er, vedrørende turneringer og sådan noget. Deltager I i turneringer? Nu ved jeg ikke hvordan turneringer foregår i floorball. Deltager i?
\\\\
Jonas: Vi har tre herre-hold tilmeldt Danmarksturneringen og ét damehold, og i år kun ét U13-hold. Sidste år havde vi både U11, U13 og U15. Men vi har så også noget i Nordjylland der hedder "kidz liga", som er for U11 og ned.
\\\\
Barb: Nårh ja, det kender jeg godt.
\\\\
Jonas: Kender du floorball?
\\\\
Barb: Nej, men det med "kids" konceptet, fordi kidsvolley er også... [en ting]
\\\\
På samme tid: Ja.
\\\\
Jonas: Det var jo det jeg faktisk var til i weekenden i Frederikshavn, hvor vi så startede ud med U6'erne, ej det hed U5 der men, det var dem vi spillede med. U5, U7, U9 og U11, hvor de så spiller i omkring en halv times tid hver årgang på fire baner inde i en hal, og bare for at spille en hel masse floorball. Så kommer den næste årgang ind bagefter. Så på den måde kan vi få rigtig mange børn igennem på en dag. 
\\\\
20:02
\\\\
Barb: Kan det passe at de ikke får point på samme måde?
\\\\
Barb: [Uforståeligt]
\\\\
Jonas: Alle kampene ender tre-tre.
\\\\
Barb: Ja, så okay. Så er det lige præcis sådan det er.
\\\\
/[Lidt forvirring]/
\\\\
Jonas: Jo, vi har en del kampe og afvikle. Jeg tror jeg programsatte lige godt hundrede kampe i sommers eller sådan noget.
\\\\
Barb: Hundrede hvad?
\\\\
Jonas: Hundrede kampe. Altså hjemmekampe. Kan det være rigtigt? Det virker som mange. 
\\\\
Barb: Det gør det.
\\\\
Jonas: 50-60 måske
\\\\
20:40
\\\\
Barb: Har I så forskellige slags af turneringer? Altså f.eks. landsturnering, pokalturnering eller sådan nogen én-dags turnering. Deltager I i dem?
\\\\
Jonas: Ja, altså for ungdom er det sådan meget de her én-dags som er cirka én gang om måneden. Både vores U13 hold og vores senior hold, de deltager i Danmarksturneringen, som er landsdækkende, men delt op i regioner. I øjeblikket der spiller alle vores hold i region Nordjylland. Sidste år havde vi et hold i første division, som er delt op i øst og vest, så der var vi i Odense og Esbjerg. 
\\\\
21:25
\\\\
Barb: Har i så, altså, har i så selv de her én-dags turneringer? Arrangerer i nogen her, hvor andre hold kommer til jer?
\\\\
Jonas: Ja, det gør vi.
\\\\
Barb: Og der bruger i den samme hal og faciliteter?
\\\\
Jonas: Ja.
\\\\
Barb: I får måske mere tid fra bestyrelsen.... [Uforståeligt]
\\\\
Jonas: Ja, men altså, hvis hallen er ledig, så kan vi jo sagtens booke den en hel dag, for et eksempel. Vi har vidst siden april hvornår vi skulle holde de her stævner, og vi havde et her den 31. oktober tror jeg det var. Lyder det rigtigt? 
\\\\
Barb: Så i ved det fra begyndelsen af, så i kan aftale... [Uforståeligt]
\\\\
Jonas: Ja, vi har vidst det, ja.
\\\\
Jonas: Og vi plejer også og holde et eller to stævner derudaf, i løbet af året, altså seniorstævner.
\\\\
Barb: Også som én-dags...?
\\\\
Jonas: Ja, som nogle én-dags stævner, ja. Men eller så ligger vores, når vi har hjemmekampe, så typisk så har vi måske tre, fire eller måske endda fem kampe samme dag. Det er jo mest optimalt når vi har bander der skal sættes op, så er det rart at kunne afvikle det. Plus, det er også med til at give en hel masse klubånd, at holdene spiller efter hinanden.
\\\\
Barb: Og når I så arrangerer en turnering her. Nu ved jeg ikke om du har været med, men altså hvordan sætter i hold sammen?
\\\\
/[Forvirring om hvordan spørgsmålet skal tolkes]/
\\\\
23:43
\\\\
Jonas: Altså til stævnedage? 
\\\\
Barb: Ja.
\\\\
Jonas: Jamen det kommer jo, årh det kommer godt nok gevaldigt an på hvad... Jeg lavede faktisk programmet da vi holdte kidz liga nu her. Der kunne de tilmeldte foreninger, tilmelde hold i de forskellige rækker, altså forskellige alders-rækker. Og når vi så vidste hvor der var, der var f.eks. seks hold i U5 og sytten i U7 og nitten i U9, tror jeg, sådan noget lignende. Så er det med at finde ud hvordan deler vi det bedst ud på de tre eller fire baner vi har og spille på. Der vi ikke flere hold, end at vi godt kunne klare tre baner, og så deler vi bare de puljer op f.eks. U17 med sytten hold i hver [Ud fra konteksten om kidz liga, er det her nok en talefejl, og han mener i stedet U7 med sytten hold], dem måtte vi jo så dele ud i tre puljer, en med fem og to med seks. Og så spillede de mod hinanden, alle mod alle.
\\\\
Barb: Ja. Så du laver det næsten bare i hånden
\\\\
Jonas: Som udgangspunkt, i hvert fald i den overordnede struktur, må man lige se på, hvad har vi med at gøre. Men så brugte jeg faktisk Excel. Jeg har nogle excel-ark som er... Kode hedder det vel ikke, men de er i hvert fald sat op til at man bare kan lægge holdene ind, og så kører der et kampprogram ud fra det.
\\\\
Barb: Som en skabelon?
\\\\
Jonas: Ja, en skabelon. MEN, det ville være rart at have et ordenligt... [program]. Jeg ved at der eksisterer nogle apps til det. Men det ville være rart at have nogen man faktisk kunne bruge til sådan noget... [Uforståeligt]. Nudge, nudge.
\\\\
Jonas: Jeg har også lavet en masse skoleturneringer, altså hvor vi er ude på skoler og så laver et stævne for flere skoler sammen, eller også bare hvis der er flere spor på en skole, altså der er to eller tre klasser i samme årgang. Så laver vi en turnering eller et stævne med de hold, altså med fire på hvert hold f.eks. Og det er jo nemt nok at lave, sådan så siger vi "vi laver lige hurtigt et kampprogram". Det ville være endnu nemmere hvis man havde noget man kunne gå ind og mingelere lidt med. 
\\\\
Barb: Det tager måske lidt lang tid med så mange hold?
\\\\
Jonas: Ikke når først man har gjort det nogen gange, men hvis man så skal over og gøre det præsentabelt, hive det fra et excel-ark over... Det tager noget tid. Der kan også være nogle ting der er uhensigtsmæssige i den måde de nu er blevet fordelt på. Et excel-ark tager typisk ikke højde for om de kommer til at spille tre kampe i træk eller... Ja. Så der kan være nogle ting der. Men når man har gjort det nogle gange så er det jo... Alting bliver jo lettere. Men jeg ved at der er mange der døjer med lige nøjagtigt det der. Man kan sidde og stirre sig blind på nogle ting, og så alligevel lave nogle fejl i kampprogrammet. 
\\\\
Barb: Det kan godt ske ja.
\\\\
Jonas: Da vi var af sted her i søndags havde vi to hold med i U5 rækken der. Og jeg undrede mig noget over det kampprogram jeg fik tilsendt, og det var ikke blevet ændret da vi kom derop. Det ene af holdene havde seks kampe, og det andet have fire. Om det var en fejl fra deres side, det ved jeg ikke, men det var i hvert fald... Det tyder på at de har siddet og lavet det i hånden, fordi de [holdene] hed jo "Aalborg Flyers 1" og "Aalborg Flyers 2", så de er jo formodentlig bare kommet til at tage det ene hold en gang for meget. Men det ville aldrig være sket hvis de havde et program at smække det ind i, eller gjorde det i et excel-ark.
\\\\
27:38
\\\\
Barb: Jeg tænker også hvis det er nogen der ikke møder op til selve kampdagen, og måske ikke har sagt, så skal man jo ligesom....
\\\\
Jonas: Det har jeg hørt om.
\\\\
Barb: Jeg har også selv været til turnering hvor jeg skal lige omskrive alle kampene fordi der er ét hold der ikke sagde fra [Altså ikke meldte afbud]. Så det er lige det, og så bruge tid på det.
\\\\
Jonas: Skæbnen vil, at i morgen der holder vi det der hedder "Skolernes floorball dag", som er landsdækkende hvor man centralt har kunne tilmelde sig over hele landet. Så ligger den så inde ved de lokale klubber og afvikle med de skoler der nu har tilmeldt sig. Jeg har været ude for, jeg tror endda at det var sidste år, hvor dem der så havde tilmeldt sig, de begyndte at melde sig fra. Nogen af dem i relativt god tid, andre inden for den sidste uge inden, og der var sågar et af holdene, altså en af skolerne, der bare meldte fra lige da vi skulle til at starte mere eller mindre, eller også ringede jeg til dem for at høre hvor de.... Ej jeg tror de skrev at de ikke kom alligevel.
\\\\
Barb: Ja. Det er måske det dårlige ved at lave det i hånden. Altså det kunne måske være nemmere at have noget at taste ind, så er vi klar.
\\\\
Jonas: Altså lige nøjagtigt det der, det allerbedste vil jo være at de ikke meldte sig fra.
\\\\
Barb: Ja, lige præcis.
\\\\
% 28:57
% \\\\
% Barb: Og har i en særlige måde melde... Altså hvis i har en turnering f.eks. hvordan laver i så for de andre at melde sig til den her turnering?
% \\\\
% Jonas: Det er ofte på Facebook man gør reklame for det. Vi har nogle grupper på Facebook, og typisk så vil der måske også blive sendt en mail rundt til de enkelte foreninger. Den kommer så måske ikke nødvendigvis videre fra formanden, så derfor er det meget godt at tage kontakt til de relevante hold også. Så har vores specialforbund også gjort nogle forsøg på at lave en overordnet kalender over hvad der ligger af stævner på landsplan, men jeg tror den er sat lidt i bero lige nu, fordi de døjer lidt med nogle ting. Der har været store omvæltninger nu her inden for det sidste halve år, så nogle ting fungerer ikke helt lige i øjeblikket  
30:37
\\\\
Barb: Til turneringerne og alt det der, det har du jo svaret på. Men hvor lang tid har i til planlægning, altså hvor lang tid bruger i på det?
\\\\
Jonas: Det burde jo være rimelig simpelt. Vi får allerede et foreløbigt... Altså, til enkelt-dags turneringer eller helårs?
\\\\
Barb: Jeg tænker måske til enkeltdags? 
\\\\
Jonas: Jeg brugte måske et par timer sidst. Længere tid behøver det ikke tage. Problemet det opstår når der kommer nogle ændringer, især hvis man allerede har kommunikeret et kampprogram ud, så kan det godt blive lidt... Ja.
\\\\
Jonas: Jeg bruger meget mere tid på selve Danmarksturneringen.
\\\\
Barb: Selve hvad?
\\\\
Jonas: På Danmarksturneringen, altså på de programsatte kampe. Der får vi et foreløbigt kampprogram om sommeren, faktisk er den nogle gange allerede i maj måned, men ellers i juni, og så kan vi begynde og booke haltider og fastlægge vores kampe. Vi får typisk en weekend den skal ligge i. Og det man så er begyndt på, det er 
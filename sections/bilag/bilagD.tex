\chapter{Interview: AKG gymnastik}\label{ch:appDlabel}
Interviewere: Liv Holm og Michelle Volf Terpling\\
Interviewperson: Ina Breum\\
\\\\
størrelse af klub
\\\\
udstyr 3:31 - 8:05
[3:31]
Liv: jo hvad med, hvad med jeres sådan, altså udstyret her? Er det skolens eller hvordan er det, det fungerer med?
\\\\
Ina: Det er faktisk vores det meste af det. 
\\\\
Michelle: Okay.
\\\\
Ina: Jeg ved ikke med de gamle heste der står derhenne. Om det lige er os der købt dem, eller om det er skolen. Men inden i det andet rum, der er der en masse skumredskaber og en airtrack, og det er faktisk nogle ting vi har købt. 
\\\\
% Hvad er en airtrack
[...]
 \\\\
[4:27]
Liv: Så skolen bruger ikke jeres udstyr.
\\\\
Ina: Jo. 
\\\\
Liv: Det gør de også?
\\\\
Ina: Jo. Og det kan godt give udfordringer. 
\\\\
Michelle: Okay.
\\\\
Ina: Fordi det, altså her er det jo skole, og der er jo også Dussen herovre. derovre, den vej jeg kom fra der er der Dus, og så er der jo så også noget klub her. 
\\\\
Liv: Okay
\\\\
Ina: Skole Dus der har noget for de ældre om aften. 
\\\\
% Hvad er en dus
[...]
\\\\
[5:05]
Michelle: Okay men så de bruger også det hele. Er det sådan aftalt på forhånd eller nej?
\\\\
Ina: Altså jeg vil sige vores airtrack der, altså vi har et skab derinde, og pumpen og sådan noget det ligger derinde. Og så fandt vi ud af at der åbenbart er nogen der har haft nøgle til det skab og bruger airtracken. Fordi så kommer man derud gangen efter, og tænkte jeg brugte ikke airtrack sidste gang, hvorfor er den rullet så grimt sammen?
\\\\
Michelle: Okay. Det lyder træls.
\\\\
Ina: Ja, nu kom jeg lige fra hvad spørgsmålet var?
\\\\
[...]
\\\\
Liv: Det var egentlig bare om i delte det med skolen?
\\\\
[5:42]
Ina: Altså så airtracken den deler vi ikke. Det var det jeg ville frem til. 
\\\\
Liv: Det er ikke meningen den skal blive brugt af de andre?
\\\\
Ina: Nej. Men de vil gerne bruge den, og de er interesserede i den. Men vi har også, vi havde et møde i foråret, nej lige inden sommer faktisk. 
\\\\
% Knirken og relateret snak. Aner ikke hvad er foregår
[...]
\\\\
Ina: Vi havde et møde inden sommerferien fordi skolen er selvfølgelig, de vil jo gerne bruge den, fordi det er jo fedt. Altså det er jo en lidt dyr investering ikke. Så vi har en aftale om at de har en nøgle til vores skab, og så er det ligesom viceværten, man skal komme og hente nøglen. Sådan at han ved hvem der ligesom har brugt det, hvis nu vi kommer en dag og altså...
\\\\
Liv: Og den ikke er blevet brugt ordenligt eller hvad?
\\\\
Ina: Ja den, fordi hvis man har sko på det ødelægger den, og der er sådan nogle propper vi skifter en gang imellem fordi de også bliver ødelagte. 
\\\\
% En masse ja'er 
[...]
\\\\
[6:41]
Liv: Så det kan godt være en udfordring med delte redskaber?
\\\\
Ina: Ja.
\\\\
Liv: Ja for det er vel jer der skal have købt det nye ind hvis det.
\\\\
Ina: Jo, altså det er jo ligesom os der har købt det, men vi siger okay i må gerne bruge det. Men det skal ligesom også behandles ordenligt. For hvis det går i stykker, vi ved jo ikke om det er skolen, eller om det er dussen, eller om det er klubben der har ødelagt det. 
\\\\
[7:06]
Michelle: Hvem er det der har ansvaret for, altså at holde styr på udstyret og sådan nogle ting. 
\\\\
Ina: Det har vi selv tænker jeg.
\\\\
Michelle: Men er det sådan, er der én person i bestyrelsen eller et eller andet der står for det eller?
\\\\
Ina: Nej, det er sådan en fælles opgave. 
\\\\
Michelle: Hvordan koordinerer i det så? og sådan med hvis der skal købes nyt ind, eller noget er slidt. Eller som du nævner, der er nogle propper, som der skulle skiftes en gang imellem.
\\\\
Ina: Altså de små ting. Altså vi er jo en bestyrelse og så. Altså det er for det meste bare en forespørgsel, eller hvis nu jeg går derinde og tænker, det er irriterende der mangler der mangler det. Jeg har lige købt hula-hop ringe faktisk, og det er også meget spændende om de ringe de bliver her. Men det er bare sådan, det vil jeg gerne have, og så stiller man ligesom på bestyrelsen, har vi råd til det. og så snakker vi om det og bliver enige om at ja det kan vi godt købe. Eller nej, det skal vi lige vente med. 
\\\\
[...]

\\\\
%[11.53]
Liv: Hvordan er det så i jeres bestyrelse? Hvor mange medlemmer er i og hvordan organiserer i jer? Altså, er det igennem Facebook vi snakker eller en masse møder i holder, og sådan?
\\\\
Ina: Ja, vi har møder ca hver anden måned. Og vi har en formand og så har vi vores kasserer og så har vi tre bestyrelsesmedlemmer. To som ligesom er der to år og så en der på genvalg hvert år, så vi er fem i bestyrelsen. og så har vi to suppleanter også,  hvor den ene suppleant, hun er ikke på Facebook. eller det tror jeg faktisk to af dem ikke er, men det er også sådan ... Så vores kommunikation er oftest email, hvis det er sådan noget der skal ud til alle. Men jeg har så oprette en Facebook gruppe til bestyrelsen, til de der "hvornår var det lige vi aftalte" eller "er der nogen der lige" du ved. Så de der lidt hurtige spørgsmål folk måske lige kan svare på.
\\\\
%[13:00]
Liv: Har i andre end jer der er i bestyrelsen? Altså har i andre frivillige der går og hjælper jer, sådan med noget i dagligdagen eller...?
\\\\
Ina; Nej, altså vi har jo bestyrelsen  og så  har vi vores instruktører, som også er frivillige, men det er jo for det meste hun på holdene, så  der er ikke så meget mere. Så  vi er ikke så stor en forening, så hvis der er noget så træder vi selv til. Altså  vi har, vi sidder ude i et, hvad hedder sådan noget, Foreningshuset. Hvor der sidder nogle andre foreninger, så der er selvfølgelig nogen der skal hjælpe til derude. Det er vores suppleant og en faktisk måske vi har en frivillig der hjælper hende med, der er en gang om året så skal de tælle op på lageret eller vaske gulv eller sådan. Der er sådan  med de ældre, der er sådan et godt sammenspil så det finder de ud af. 
\\\\
%[13:58]
Liv: Nu hvor man har gymnasitk så har man jo selvfølgelig ikke sådan turneringer på samme måde som håndbold og sådan noget, men er i nogensinde til opvisning eller hvordan fungerer det?
\\\\
Ina: Ja altså vi har altid den der årlige opvisning. den har vi så ikke haft de sidste to år fordi vi ikke har haft springhold, men ellers så har vi den årlige opvisning ude i Øsetallehallen. 
\\\\
%[14:23]
Michelle: Er det bare jer eller er det sammen med andre foreninger også?
\\\\
Ina: Det er bare os. Og så har vi vores fastelavnsfest som vi holder her hvert år til fastelavn ca deromkring. Og så har vi laver lidt noget nyt som heder søndags sjov, hvor vi stiller en redskabsbane op herinde, laver en kæmpe bane og så er det sådan halvanden time hvor forældrene bare kan komme med deres børn og så betale en tyver og så kan de lege her, uden at  der er et decideret program. 
\\\\
Michelle: Så det er bare, det er bare et legeland.
\\\\
Ina: Ja, nemlig. det er lidt det vi prøver, bare kan komme og man behøver ikke være medlem og sådan noget. \\\\
Michelle: Okay, og det er også en god måde at få nye medlemmer.
\\\\
Ina: Ja, det er det nemlig. "Tag naboen med herover og leg. Kom. Kom"
\\\\
[Lidt utydelig snakken frem og tilbage]
\\\\
Michelle: Men i laver ikke nogen opvisninger sammen med andre foreninger, altså sådan større... ud og lave et eller andet?
\\\\
Ina: NEj, nej, nej.
\\\\
Michelle: Det er ikke jeres fokus som...?
\\\\
Ina: Nej, jeg tror måske som, helt tilbage i tid, har man måske gjort det et par gange, men det gør vi ikke mere.
%[15;53]
Liv: Og i brugte ikke nogen  andre planlægningsværktøjer end Facebook og mails? Der er ikke sådan noget elektronisk i bruger?
\\\\
Ina: Nu skal jeg lige se... nej, altså der er jo selvfølgelig, der er Aalborgs bookingsystem til at booke lokaler og så Facebook til at kommunikere, det er både internt og udadtil vi bruger det. Og så har vi så Nemtilmeld også i forhold til at håndtere tilmeldinger til holdene og arrangementer. 
\\\\
%[16:28]
Michelle: Og hvis der var et eller andet der manglede eller man ville lave et nyt arrangement, som det Søndags sjov. Så er det bare noget man foreslår for bestyrelsen altså, der er ikke sådan en proces, en større proces, for at komme til det eller et system i det. Det er bare sådan lidt spontant?
\\\\
Ina: Altså tænker i til søndags sjov altså...?
\\\\
Michelle: Ja, ja, eller hvad hedder det, med nye ting, hvis man vil lave et nyt arrangement eller hvis der mangler noget?
\\\\
Ina: Nej, det plejer bare at være, du ved altså, mail. Altså vi har også fået, vi har fået en ny instruktør i år, så hvis de, de har også foreslået nogle nyindkøbte ting eller et kursus de gerne vil på og så. det foregår oftest på mail eller hvis man lige, de... Stramopholdet de kommer så efter mig om tirsdagen. Så fanger man lige hinanden der. 
\\\\
%[19:26]
Liv: Men altså hvor ofte bliver jeres ting slidt ned, sådan med udstyret?
\\\\
Ina: Altså jeg synes jo at det er noget gammelt vi har. Sidste sommer der fik jeg købt en masse nye skumredskaber, fordi der ikke var noget. Der var mest sådan nogle store ting som man bruger til spring, men der var ikke sådan noget til, til de helt små. 




%frivillige 11:53 - 12:59
%(8:49 - mangel på frivillige) (10:41 - hvor mange og hvilke frivillige)
%bestyrelsesmøde 16:28 - 17:31, 19:28 - 20:05

%ingen turneringer 14:01 - 15:33
%ingen værktøjer til planlægning 15:52 - 16:28
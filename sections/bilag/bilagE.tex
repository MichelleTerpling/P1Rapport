\chapter{Interview: Aalborg karate skole}\label{ch:appElabel}
Interviewere: Ane Søgaard Jørgensen og Michelle Volf Terpling\\
Interviewperson: Mogens Juul Møller
\\\\
Udstyr:
[00:11]
Mog: Det var omkring Aalborg karateskole. Og så udstyr i forbindelse med at dyrke karate.
\\\\
Ane: Ja
\\\\
Mog: Det jo typisk personlige ting man har. Altså man har en gi, en dragt som man træner i. Hvis man spiller håndbold så har man også sådan noget håndboldtøj på, så har man også en karate gi, og så har man et bælte. Det er egentlig det man har, og så hvis man stiller op til kamp turnering eller et andet i det der hedder kumite, hvor man kæmper mod hinanden, så har man mere beskyttelsesudstyr på. 
\\\\
Ane: Er det så foreningens eller er det også personens? 
\\\\
Mog: Det er personlige. det er jo handsker, fodbeskyttelse, vest osv.
\\\\
...
[00:58]
Mog: Og så har vi noget inde i klubben til dem som måske ikke har været ude og kæmpe. Det er så ikke personligt, der kan man gå ind og tage. Det står bare i en stor snavstaskekurv eller sådan noget i den stil som vi opbevarer, så går man ind og tager det man har brug for så er der nogle sparkepuder og sådan nogle ting. Så har vi selvfølgeligt nogle måtter i klubben, men det er ikke noget som man flytter rundt med. Det er så for at skåne led og sådan nogle ting, når man træner.
\\\\
Så der er heller ikke ret meget udstyr man har der til.
Og så har man lidt dommerudstyr, men det er ikke sådan som vi går og flytter rundt. Det er et ur, en gong gong, og så vender man nogle tal. Så det er relativt simpelt, det man har har måske heller ikke en så stor værdi. 
Vores måtter er jo ret dyre ikke, men de ligger sådan set fast så blive de kun løftet op engang imellem når der skal gøres rent nedenunder dem.
\\\\
Mich: Men har I, jeg ved ikke hvor mange der er i bestyrelsen som frivillige i foreningen...
\\\\
Mog: Vi er fem i bestyrelsen
\\\\
Mich: Er der noget, med hensyn til ansvar [uforståeligt].. 
Det udstyr I har, er der en der sådan står for det?
\\\\
Mog: Nej, vi har ikke sådan nogle udstyrsansvarlige, vi hjælper lidt med at sælge(?) 
\\\\
Turneringer:
[7:41] - 
nævner kata ca. [9:30]
\\\\
[9:59]
Mogens: Og så den anden det er, kata det sådan, der laver man nogle øvelser. Du kan sige sådan et forløb man skal gå igennem. så når man har gjort det, så laver den anden også en kata, og så bedømmer man hvem der laver den teknik og præcision og balance og styrke og så videre, bedst. så finder man i forhold til at gå videre på den måde.
\\\\
Michelle: Okay, okay så der er man slet ikke oppe mod hinanden? det mere sådan
\\\\
[10:29]
Mogens: Jo det er man så også, men ikke i kontakt med hinanden.
\\\\
Michelle: ikke fysisk.
\\\\
Ane: Man er imod sig selv, på en eller anden måde.
\\\\
[10:34]
Mogens: ja det er mod sig selv, kan man sige. Men så modstanderen, modstanderen skal også ind og lave en kata. Det behøver ikke at være den samme man konkurrer med, men der er jo kata i forskellige sværhedsgrader, så det bedømmer dommerne så, hvem der klarer den bedst kan man sige. [...]
\\\\
[19:42]
Michelle: Hvor mange er i sådan, i alt cirka?
\\\\
Mogens: Jeg tror vi er 150 eller sådan noget i alt. 
\\\\
Ane: Altså medlemmer?
\\\\
Mogens: Ja. 
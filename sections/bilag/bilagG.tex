\chapter{Projektforslag: \\Effektivisering i landbruget}\label{ch:appGlabel}
%\section*{Effektivisering i landbruget}
\textbf{Problemstilling}\\
Landbrugssektoren i Danmark er blevet meget effektiviseret og der bliver stillet store krav til den enkelte landmand. Det historiske dårlige vejr i de sidste to år, har været med til at skabe store tab for nogle landmænd. Der skal således optimeres der hvor der kan, for at ændre tab til profit for landbrugene. Dette kunne hjælpes på vej ved at effektiviseres på både små og større processer i landbruget. En landmand bruger allerede det meste af dagen på at arbejde og derfor er svaret at effektivisere landbruget i stedet.

\textbf{Mål}\\
Målet er at effektivisere et landbrug i Danmark ud fra at analysere og forstå deres behov. På baggrund af denne analyse skal der findes et velafgrænset, velargumenteret og konkret problem, som skal løses gennem en software løsning.

\textbf{Eksempel på datalogiske problemstillinger}\\
Et eksempel på en datalogisk problemstilling ville være at udvikle et planlægningssystem, som kunne løse en landmands konkrete problem. Der findes allerede mange planlægningssystemer, som bruges til andre områder. Man kunne få inspiration fra disse til løsning af en problemstilling inden for effektivisering i landbruget.  

\textbf{Eksempel på kontekstuelle spørgsmål og problemstillinger}\\
Dette projektforslag er placeret i kontekst af små landbrug, hvor det netop er svært at få økonomien til at køre rundt pga. De mange ting der skal laves på en dag. Kan der laves et simpelt IT-system, som kan støtte et mindre landbrug i at bliver mere effektiv.  



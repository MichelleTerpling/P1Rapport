\chapter{Konklusion}\label{ch:conclusion}
% Programmet kører *knipslyd* sådan der.

Dette projekt startede med en initierende problemstilling om ressourcefordeling i idrætsforeninger. For at finde frem til et afgrænset problem inden for dette, blev der udført interviews med personer fra forskellige idrætsforeninger, der havde indsigt i de problemer foreningen havde. Dette endte med det afgrænsede problem at det er en udfordring at lave og ændre kampplaner, til kidzliga stævner i floorball. 
\\
Herefter blev der designet, og efterfølgende implementeret en løsning til dette problem. Løsningen er også blevet vurderet, i forhold til om den har overhold de krav der blev sat op. 
På baggrund af alt dette laves der i dette afsnit en konklusion på, om problemstillingen er blevet løst.
\\\\
Målet med dette projekt, er at gøre det muligt at lave en stævneplan hurtigere og med færre fejl, end det kan gøres manuelt. Desuden var det også målet at gøre det lettere og hurtigere at rette en allerede færdig stævneplan. 
\\\\
% Afsnit om en lille test jeg lavede, ved ikke om det skal med. Kunne flyttes til andet afsnit, eller et bilag
Det antages at programmet er i stand til at lave en stævneplan hurtigere end et menneske. På trods af at der ikke er foretaget nogen test i dette projekt, der afgør hvor hurtigt et menneske kan gøre det. Men der er blevet foretaget tests af hvor lang tid visse funktioner i programmet tager. 
Testen, der kan findes i Kildekode \ref{code:timeTest}, er foretaget på funktionen \textbf{\textit{createNewTournament}}. Da visse dele af denne funktion er afhængig af brugeren, og dermed dennes tidsforbrug, er de ikke blevet testet. Den del af processen der er testet er fra linje 20 i kildekode \ref{code:createNewTournament} på side \pageref{code:createNewTournament}, og frem til og med kaldet af \textbf{\textit{printingMenu}} i linje 60. Det bemærkes at hver test blev foretaget fem gange og det længste tidsforbrug er præsenteret her. Denne del af funktionen tager 0,306 sekunder.
Det er også testet hvor langt tid funktionen \textbf{\textit{printProgram}}, tager om at skrive stævneplanen ud til terminalen. Denne proces tog 0,195 sekunder.
\\
I en virkelig situation skal der selvfølgelig tages højde for, at brugeren skal bruge tid til at indtaste indformation. Men det antages at selv med dette yderligere tidforbrug lagt til, vil programmet være hurtigere til at udarbejde en stævneplan.

\begin{listing}
\begin{minted}[frame=lines, framesep=3mm, baselinestretch=1, linenos, bgcolor=LightGray]{c}

clock_t begin = clock();

/* Proces der testes */

clock_t end = clock();
double time_spent = (double)(end - begin) / CLOCKS_PER_SEC;

\end{minted}
\captionof{listing}{Koden bruges til at måle tidsforbruget af en proces. Dette gøres ved at måle antallet af clock-ticks der bruges af en proces, og ved brug af \textbf{\textit{CLOCKS\_PER\_SEC}} omregnes dette til sekunder. Dette kodeeksempel bruger standardbiblioteket time.h}
\label{code:timeTest}
\end{listing}

Det er også et krav at stævneplanen skal have færre fejl, end en plan der er lavet manuelt. Dog kan en stævneplan selvfølelig ikke have færre end nul fejl, og derfor ikke være bedre end en manuelt udarbejdet stævneplan uden nogle fejl. 
Dette krav opfyldes som konsekvens af programmet fremgangsmåde. Programmet sammensætter en tilfældig plan, runde efter runde. Hvis den udarbejder en plan der bryder nogle af de opsatte regler, forsøger programmet at lave rettelser. men hvis det ikke er muligt at korrigerer planen, bliver der istedet udviklet en helt ny plan fra bunden. Derfor vil det resulterende program, aldrig bryde med nogle af de hårde regler der blev sat som krav. 
\\\\
I problemformuleringen stilles der også krav til at en stævneplan skal være mere fleksibelt. Altså skal det være muligt at redigere planen. Dette bliver specificeret yderligere i kravene til programmet, hvor der står at det skal være muligt at tilføje og fjerne hold.
Dette krav opfyldes ved at give en bruger mulighed for at tilgå en menu, hvorfra det er muligt at fjerne eller tilføje ét eller flere hold. I denne menu er der også mulighed for at lave en helt ny stævneplan, hvis der ønskes en ny, uden at ændre på antallet af deltagende hold.
\\
Den stævneplan der bliver udarbejdet efter redigering vil blive opstillet tilfældigt. Hvilket betyder at en stævneplan der er et resultat af redigering, ikke nødvendigvis vil ligne den originale.
\\\\
% Kort opsummering
Løsningen er altså i stand til at lave stævneplaner, til kidzliga-stævner, hurtigt og uden fejl. Denne plan er også fleksibel da der er mulig for at lave ændringer hurtigt. Det er også muligt at videre udvikle dette program til at tage hensyn til flere former for turneringer og forskellige sportsgrene.



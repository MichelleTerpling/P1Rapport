\chapter{Indledning}\label{ch:introduction}
I 2017 var der 11,635 idrætsforeninger, samlet i DGI, DIF og Firmaidrætten, med i alt 2,552,090 medlemmer og 539,821 frivillige\cite{fester2018}.
\\
% Danmark har længe haft et velfungerende foreningsliv. Helt tilbage fra 1700-tallet[XXX] er folk kommet sammen med andre ligesindede, om..... (I 1800-tallet var det gud, i 1700-tallet var det patriotisk).

% Handler om hvorfor ressourcefordeling er et problem, skrive overordnet.

% Beskrivende del omkring ressourcefordelings problemer i sportsklubber.

% Det hele består af frivilliges hjælp, og derfor skal ting fungere overskueligt og nemt for de frivillige, da de ikke får nogen løn.

% Det er en god ide at hjælpe da der er så mange medlemmer i de danske sports foreninger og man for så mange gode ting ud af at dyrke sport i en forening.

% Beskriv lidt mere om det problem vi har valgt at gå i dybden med.

% Det brede kig på hvorfor det overhovedet er interessant at kigge op sportsforeninger. Det rammer mennesker (medlemmer og frivillige). Find tal og andet godt.
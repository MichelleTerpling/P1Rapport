\chapter{Indledning}\label{ch:introduction}
Den første danske idrætsforening blev stiftet i 1861 \cite{difhistorie}.
\\
\\
Mange danskere er en del af en idrætsforening eller har været det på et tidspunkt i deres liv. I 2017 var der 11.635 idrætsforeninger, samlet i DGI, DIF og Firmaidrætten, med i alt 2.552.090 medlemmer og 539.821 frivillige \cite{fester2018}. Når der er så mange mennesker i idrætsforeningerne kan der nemt opstå problemer med organiseringen af idrætsforeningers aktiviteter og ressourcer. 
\\
\\
% Hvordan hænger en idrætsforening sammen?
En idrætsforening består af en samling frivillige, sammenholdt af en række vedtægter, et fælles formål og evt. et sæt værdier. Foreningnen er demokratisk opbygget. Ved generalforsamlinger tages beslutninger for foreningen, bl.a. hvem der sidder i bestyrelsen. Mellem generalforsamlingerne er det bestyrelsen der leder foreningen og har bl.a. ansvaret for at sikre at vedtægterne følges, at foreningen arbejder mod sit mål, samt at have styr på økonomien. Ud over bestyrelsen, er der også andre frivillige der fx fungerer som trænere [xxx].
% Noget om hvad der skal til for at få økonomisk støtte af kommunen
% https://www.dgi.dk/foreningsledelse/start-forening/kom-godt-i-gang/guide-saadan-opretter-i-en-forening
% https://frivillighed.dk/guides/bestyrelsens-opgaver
\\
\\
Frivillige er fundamentet i idrætsforeningen, og uden dem falder det hele fra hinanden. Derfor kan det være til gavn for foreningerne at gøre trivielle opgaver, som frivillige i en forening til hverdag skal løse, lettere eller fjerne dem helt. Dette vil give de frivillige mere tid til at fokusere på vigtigere opgaver, såsom børnene.
Denne løsning kunne komme i form af et computerprogram. (skal udvides)
\\
\\
En gennemsnitlig forening der hører under hovedorganisationen DIF havde i 2017 ca. 215 medlemmer. Hvis udstyret i en sådan forening bliver brugt regelmæssigt, er det derfor rimeligt at antage at udstyret vil blive slidt. \citep{idrætTal2017}
Hvis der ikke bliver hold styr på slidt udstyr, sådan at det kan blive udskiftet når det bliver nødvendigt, kan en forening ende i den situation at de bliver nødt til at bruge slidt udstyr.
\\
\\
Der er problemer med rekruttering...  % Skal det overhovedet med her? Der er en kilde på dette
\\
\\
I dette projekt vil der derfor blive undersøgt, hvordan man ved hjælp af et program kan løse problemet med at der ikke er tilstrækkeligt overblik over slidt udstyr i idrætsforeninger. .... % Hvilket problem løser vi?
\\
\\
Det initierende problem for dette projekt er ressourcefordeling, og specielt organisering af disse ressourcer. 
\\
\\


% Handler om hvorfor ressourcefordeling er et problem, skrive overordnet.

% Beskrivende del omkring ressourcefordelings problemer i sportsklubber.

\chapter{Indledning}\label{ch:introduction}
Den første danske idrætsforening der fik støtte fra staten blev stiftet i 1861 \cite{difhistorie}.\\
Siden da har foreningslivet blomstret, og mange danskere er i dag en del af en idrætsforening, eller har været det på et tidspunkt i deres liv. I 2017 var der 11.635 idrætsforeninger, samlet i forbundene Danske Gymnastik- & Idrætsforeninger (DGI), Danmarks Idrætsforbund (DIF) og Firmaidrætten, med i alt 2.552.090 medlemmer og 539.821 frivillige \cite{fester2018}.
\\\\
En idrætsforening består af en samling frivillige, og bliver sammenholdt af en række vedtægter, et fælles formål og evt. et sæt værdier. En idrætsforening er demokratisk opbygget. Dette ses eksempelvis på generalforsamlinger, hvor der tages beslutninger for foreningen, bl.a. hvem der sidder i bestyrelsen. Mellem generalforsamlingerne er det bestyrelsen der leder foreningen og har bl.a. ansvaret for at sikre at vedtægterne følges, at foreningen arbejder mod sit mål, samt at have styr på økonomien. Ud over bestyrelsen, er der også andre frivillige, der fx fungerer som trænere, og andet nødvendigt mandskab \citep{DGI} \citep{bestyrelsen}.
% Noget om hvad der skal til for at få økonomisk støtte af kommunen
\\\\
Når der er så mange mennesker i idrætsforeningerne, kan der nemt opstå problemer med organiseringen af aktiviteter og ressourcer. Det initierende problem i dette projekt er derfor ressourcefordeling i idrætsforeninger, og specielt organisering af disse ressourcer.
Ud fra dette vil der kigges nærmere på idrætsforeninger og undersøge om der findes opgaver eller områder, som kan optimeres. Dette vil kunne gavne de frivillige, da de vil være i stand til at fokusere på andre vigtigere opgaver. 
\\\\
%En af de vigtigste ressourcer er frivillige, som er fundamentet af en idrætsforening, og uden dem falder det hele fra hinanden. Derfor kan det være til gavn for foreningerne at gøre trivielle opgaver, som frivillige i en forening til hverdag skal løse, lettere eller fjerne dem helt. Dette vil give de frivillige mere tid til at fokusere på vigtigere opgaver, såsom medlemmerne og aktiviteterne.
%\\
I dette projekt undersøges det hvilke af sådanne problemer, virkelige idrætsforeninger står over for, med det formål at finde et problem der kan løses ved brug af software.
% Alt det ovenstående udgør cirka én side, med overskriften. Overvej hvor lang indledningen skal være.

%Herunder begynder vi allerede at introducerer problemer. Er det for tidligt?
%En gennemsnitlig forening der hører under hovedorganisationen DIF havde i 2017 ca. 215 medlemmer. Hvis udstyret i en sådan forening bliver brugt regelmæssigt, er det derfor rimeligt at antage at udstyret vil blive slidt \citep{idraetTal2017}.
%Hvis der ikke bliver hold styr på slidt udstyr, sådan at det kan blive udskiftet når det bliver nødvendigt, kan en forening ende i den situation at de bliver nødt til at bruge slidt udstyr.

% Skal det overhovedet med her? Der er en kilde på dette
%Der er problemer med rekruttering...  

%I dette projekt vil der derfor blive undersøgt, hvordan man ved hjælp af et program kan løse problemet med at der ikke er tilstrækkeligt overblik over slidt udstyr i idrætsforeninger. .... % Hvilket problem løser vi?


% Handler om hvorfor ressourcefordeling er et problem, skrive overordnet.

% Beskrivende del omkring ressourcefordelings problemer i sportsklubber.


% Telefonnummer: 28777563
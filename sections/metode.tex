\chapter{Metode}\label{ch:ch2label}
I dette projekt er der foretaget en undersøgelse med interviews, med det formål at finde potentielle problemer indenfor ressourcefordeling i idrætsforeninger. Valget af undersøgelsesmetode er vigtigt, da forskellige metoder kan give forskellige resultater. I dette kapitel vil der blive argumenteret for valg af undersøgelsesmetoder.


\section{Undersøgelsesmetode}
% Talesprog, i know.....
Undersøgelsen i dette projekt, benytter sig af en kvalitativ induktiv fremgangsmåde. Ved starten af dette projekt, er der ikke en tilstrækkelig stor vidensbase, til at opbygge en hypotese, der kunne danne grundlag for en problemstilling. Derfor er det nødvendigt at opnå en brugbar baggrundsviden. Dette ville ikke kunne fungere med en kvantitativ metode, da vi endnu ikke kender nok til forholdene i idrætsforeningerne, til at opstille en hypotese, og derfor ikke ville kunne få meningsfyldt data tilbage fra en kvantitativ undersøgelse. I stedet vil vi interviewe mennesker, der arbejder i idrætsforeninger.
% Beskrive hvad et interview er?  - En kvalitativ undersøgelsesmetode
\\.
Ved at foretage interviews med interviewpersonen, er det muligt at stille større spørgsmål der resulterer i mere detaljerede svar end "ja", "nej", "enig" eller "uenig". Der er også mulighed for at følge op på spørgsmålene med det samme, og få svarene uddybet. Dette er ikke en mulighed med en kvantitativ fremgangsmåde, da det eksempelvis foregår igennem spørgeskemaer. Her kan kommunikation mellem intervieweren og interviewpersonen ikke foregå på en praktisk måde ud over den første gang man kontakter personen.
\\
\\
Grundet manglen på en hypotese forud for undersøgelsen, anvendes der ikke deduktiv metode. Undersøgelser i deduktiv metode  anvendes oftest til at undersøge en hypotese der er opstillet på forhånd, og ikke til at opbygge den.
\\
\\ %skal sættes til et afsnit der forklare om valg af induktiv metode.
Da deduktiv metode ikke anvendes...
Et alternativ til deduktiv metode, er induktiv metode. Denne tillader at foretage undersøgelser med formål at opbygge en hypotese.
\\
\\
Der kunne argumenteres for, at der skulle bruges kvantitativ metode, efter de forskellige interviews er gennemført. På denne måde ville det vise sig om de problemer interviewene afdækker, er generelle, eller kun gælder i den lokale idrætsforening. Dette anses dog ikke for nødvendigt i denne undersøgelse, da det er en mindre undersøgelse, som har til formål at undersøge hvorvidt der overhovedet er et problem vi kan løse.\\




\section{Interview opbygning}
Der skal stå noget her!!! \\
\\
Skriv noget kort om Lars fra Farstrup!\\
Hvad han gør i dagligdagen som fx arbejde\\
Om han er uddannet\\
Hans rolle i sportsklubben\\


\subsection {Struktur af interviewet}
I denne undersøgelse er det valgt at foretage interviewene på en semistruktureret måde. Her forberedes og gennemgås fastlagte spørgsmål, men der bibeholdes den fleksibilitet der kendetegner en almindelig samtale. Dette giver intervieweren mulighed for at gå igennem de emner, de har behov for at lære noget om, og samtidig kan der stilles uddybende spørgsmål, eller interviewpersonen kan påvirke samtalen med deres eget perspektiv \citep{brinkmann2014}.\\
Dette gøres fordi der i denne undersøgelse ledes efter et problem, der kan løses. Der er hermed brug for at kunne spørge ind til de emner, som kunne lede til identificeringen af et sådant problem. Ligeledes giver den semistrukturerede metode interviewpersonen mulighed for selv at bringe emner på banen, hvis de tror de kan være brugbare.

\subsection {Medie for interviewet}
Det er besluttet at lave interviewene ansigt til ansigt, da denne måde at interviewe på giver adgang til informationer gennem kropssprog, i modsætning til fx interviews over telefonen eller internettet. Svend Brinkmann anbefaler derfor at intervieweren selv transskriberer interviewet.\citep{brinkmann2014} I denne undersøgelse antager vi dog, at dette ikke er strengt nødvendigt, da de emner, som interviewet skal belyse, ikke er personlige for den interviewede, og således er deres kropssprog antaget til at være mindre relevant.









Hvordan interviewer vi folk? Forskningsspørgsmål og interview spørgsmål.
\\
\\
Motiver. Brinkmann side 39.
\\
\\
Semistruktureret, frem for struktureret.
\\
\\
Hvorfor antallet af interviews. Hvad ville være ideelt, og hvorfor har vi ikke det ideele.
\\
\\
Ifølge Steiner Kvale skal man interviewe det antal personer der er nødvendigt, for at opnå resultater. Derudover angiver han at antallet almindeligvis er 15 +/- 10, så vi er ikke helt hvor vi skal være, men det er også okay da det er en undersøgelse med lille omfang.
\cite{kvale2015}
\\
\\
Hvordan transkriberer vi?
\\






\chapter{Metode}\label{ch:chlabel}
I dette kapitel bliver de metoder, der anvendes i projektet, præsenteret og beskrevet. Valget af undersøgelsesmetode er vigtigt, da forskellige metoder egner sig bedst til besvarelse af forskellige typer spørgsmål. Desuden argumenters der for disse valg, samt de fravalg der er foretaget. I projektet blev det valgt at foretage undersøgelser i form af interviews med det formål at finde potentielle problemer indenfor ressourcefordeling i idrætsforeninger.

\section{Undersøgelsesmetode}
Undersøgelsen i dette projekt benytter sig af en kvalitativ induktiv fremgangsmåde. 
Ved starten af dette projekt er der ikke en tilstrækkelig stor vidensbase til at opbygge en hypotese, der kan danne grundlag for en problemstilling. Det er derfor nødvendigt at opbygge brugbar baggrundsviden. Dette vil være svært at opnå med en kvantitativ metode, da der ikke er nok kendskab til forholdene i idrætsforegningerne forud for undersøgelsen. I stedet foretages der kvalitative undersøgelser i form af interviews med frivillige medlemmer af idrætsforeninger.
\\\\
Ved at foretage interviews, frem for kvantitativ dataindsamling, er det muligt at stille større spørgsmål, der resulterer i mere detaljerede svar end "ja", "nej", "enig" eller "uenig". Interviews kan også være mere fleksible som beskrevet i næste afsnit. Dette er ikke en mulighed med en kvantitativ fremgangsmåde, hvor dataindsamlingen eksempelvis foregår igennem spørgeskemaer. Her kan kommunikation mellem intervieweren og interviewpersonen ikke foregå på en praktisk måde, ud over den første gang man kontakter personen \citep{kvale2015}.
\\\\
Ifølge Steiner Kvale skal man interviewe det antal personer, der er nødvendigt for at opnå tilstrækkelige resultater. Derudover angiver han, at antallet almindeligvis er 15 +/- 10 \cite{kvale2015}. I denne undersøgelse bedømmes fire interviews til at være tilstrækkeligt, da formålet med interviewene er at finde et problem, som projektet kan afgrænses til. Der kunne argumenteres for, at der skulle bruges kvantitativ metode, efter de forskellige interviews er gennemført. På denne måde ville det vise sig om de problemer interviewene afdækker, er generelle, eller kun gælder i den lokale idrætsforening. Hvis produktet af dette projekt skulle markedsføres, kunne der foretages en kvantitativ undersøgelse af det potentielle marked. Den kvantitative undersøgelse er dog ikke blevet foretaget i dette projekt, da denne metode ikke ville kunne bidrage til at finde et problem, hvilket er formålet med undersøgelsen.
\\\\ 
I denne undersøgelse anvendes den induktive metode. Denne metode tillader at foretage undersøgelser med det formål at opbygge en hypotese. 
I sin bog "Det kvalitative interview" skriver Svend Brinkmann om induktion:
\\\\
\say{Induktion består i at registrere en række individuelle tilfælde [...] for at sige noget generelt om den givne klasse af tilfælde.} \citep{brinkmann2014}
\\\\
Den induktive metode er ofte forbundet med den kvalitative undersøgelse og egner sig især til at undersøge nye emner, før man har viden nok til at fremsætte en hypotese. 
\\
Grundet manglen på en afgrænset hypotese forud for undersøgelsen, anvendes der således ikke en deduktiv metode i dette projekt. Deduktion anvendes oftest til at undersøge en hypotese, der er opstillet på forhånd, og ikke til at opbygge den. Man kunne eventuelt fortsætte sin induktive undersøgelse med at undersøge sin nye hypotese deduktivt \cite{deduktiv}.

% Vi bruger hermed den induktive metode til at indsamle tilstrækkelig nok viden så vi kan lave en hypotese. Dette kaldes for 'grounded theory', da der alene kigges på den kvalitative indsamlede data fremfor hvad andre forskere har sagt. Det er en længere proces, hvor materialet løbende evalueres. Man skaber teorier og hypoteser baseret på de undersøgtes erfaringer med en specifik proces.

% Ved ikke om dette afsnit ^ er nødvendigt... Det er noget med at man kigger på en hvordan folk har oplevet en bestemt proces og stiller spørgsmål til det. 

\section{Interview opbygning}
Der findes forskellige former for interview-metoder. Dette kunne fx være, hvor struktureret interviewet skal være, og måden hvorpå det skal udføres. For at opnå det ønskede mål med interviewet skal der derfor vælges struktur og transskriptionsmetode for interviewet.

\subsection {Struktur af interviewet}
I denne undersøgelse er det valgt at foretage semistrukturerede interviews ansigt til ansigt. Her forberedes og gennemgås fastlagte spørgsmål i form af en interviewguide, som udformes på baggrund af den initierende problemstilling. Guiden inddeles i forsknings- og interviewspørgsmål, så der er mulighed for at komme omkring alle emner i problemet. Her er forskningsspørgsmålene de emner der undersøges, og interviewspørgsmålene er de spørgsmål, som stilles til interviewpersonerne for at få de informationer, der mangler for at belyse emnet. Det er hermed en vejledende skabelon over de spørgsmål, som der ønskes svar på (Bilag \ref{ch:appAlabel}). På denne måde bibeholdes fleksibiliteten, der kendertegner en almindelig samtale. Dette giver intervieweren mulighed for at gå igennem de emner, de har behov for at lære mere om, og samtidig kan der stilles uddybende spørgsmål, eller interviewpersonen kan påvirke samtalen med deres eget perspektiv \citep{brinkmann2014}\citep{kvale2015}.\\
Dette gøres, fordi formålet med undersøgelsen er at finde et problem, der kan løses. Der er hermed mulighed for at kunne spørge ind til de emner, som kunne lede til identificeringen af et sådant problem. Ligeledes giver den semistrukturerede metode interviewpersonen mulighed for selv at bringe emner på banen, hvis de vurderer at de kan være relevante.

\subsection{Transskription}
De interviews, der bliver foretaget i løbet af undersøgelsen, optages på mobiltelefon. På den måde er analysen ikke afhængig af interviewerens hukommelse. Optagelsen af et interview tillader også, at det senere bliver transskriberet, hvilket gør det muligt at foretage en analyse senere. 
\\
\\
Interviewet bliver transskriberet ordret, bortset fra udtryk såsom 'øhh', 'hmm' og andre lyde eller pauser, der ikke bidrager til indholdet af samtalen. Dette resulterer i mindre fortolkning fra personen, der foretager transskriptionen, da der ikke er behov for at afkode fx interviewpersonens kropsprog. Talesprog kan være svært at forstå, når det står på skrift. Transskriptionen kan derfor blive mere udfordrerene at læse \citep{kvale2015}. Dette er grunden, til at udtrykkene og anden nonverbal kommunikation også bliver udeladt. 
\\
Det første interview er transskriberet fuldt ud for at give et eksempel på, hvordan strukturen af interviewene er. I resten af interviewene transskriberes der kun de dele, som er relevante for problemanalysen. Der er ingen standarder for transskription, og da det er afhængig af tid og ressourcer, vælger flere forskere at nøjes med at transskribere uddrag af deres interviews \citep{brinkmann2014}\citep{kvale2015}. Derfor er denne transskriberingsmetode valgt til dette projekt.
\\\\\\
I projektet benyttes en kvalitativ undersøgelse, da det giver mulighed for at få en uddybning af de spørgsmål, der ønskes svar på. Der er taget udgangpunkt i en semistruktureret form, da interview personenerne får frihed til at svare, som de ønsker og uden begrænsninger. Da der kan forekomme forvirring i forståelsen af transskriptionen af interviewet, er fyldeord blevet undladt.

\chapter{Metode}\label{ch:ch2label}
I dette projekt er der foretaget en undersøgelse med interviews, med det formål at finde potentielle problemer indenfor ressourcefordeling i idrætsforeninger. Valget af undersøgelsesmetode er vigtigt, da forskellige metoder kan give forskellige resultater. I dette kapitel vil der blive argumenteret for valg af undersøgelsesmetoder.


\section{Undersøgelsesmetode}
% Talesprog, i know.....
Undersøgelsen i dette projekt, benytter sig af en kvalitativ induktiv fremgangsmåde[XXX]. Ved starten af dette projekt, er der ikke en tilstrækkelig stor vidensbase, til at opbygge en hypotese, der kunne danne grundlag for en problemstilling. Derfor er det nødvendigt at opnå en brugbar baggrundsviden. Dette ville ikke kunne fungere med en kvantitativ metode, da vi endnu ikke kender nok til forholdene i idrætsforeningerne, til at opstille en hypotese, og derfor ikke ville kunne få meningsfyldt data tilbage fra en kvantitativ undersøgelse. I stedet foretages der interviewes med mennesker, der arbejder i idrætsforeninger.
% Beskrive hvad et interview er?  - En kvalitativ undersøgelsesmetode
\\
Ved at foretage interviews, frem for eksempelvis spørgeskemaer, er det muligt at stille større spørgsmål der resulterer i mere detaljerede svar end "ja", "nej", "enig" eller "uenig". Der er også mulighed for at følge op på spørgsmålene med det samme, og få svarene uddybet. Dette er ikke en mulighed med en kvantitativ fremgangsmåde, da det eksempelvis foregår igennem spørgeskemaer. Her kan kommunikation mellem intervieweren og interviewpersonen ikke foregå på en praktisk måde, ud over den første gang man kontakter personen. \citep{kvale2015}
\\
\\
Der kunne argumenteres for, at der skulle bruges kvantitativ metode, efter de forskellige interviews er gennemført. På denne måde ville det vise sig om de problemer interviewene afdækker, er generelle, eller kun gælder i den lokale idrætsforening. Dette anses dog ikke for nødvendigt i denne undersøgelse, da det er en mindre undersøgelse, som har til formål at undersøge hvorvidt der overhovedet er et problem vi kan løse.
\\
\\ 
I denne undersøgelse anvendes den induktive metode. Denne tillader at foretage undersøgelser med det formål at opbygge en hypotese. 
I sin bog "Det kvalitative interview" skriver Svend Brinkmann om induktion:
\\
\\
\say{Induktion består i at registrere en række individuelle tilfælde [...] for at sige noget generelt om den givne klasse af tilfælde.} \citep{brinkmann2014}
\\
\\
Den induktive metode er ofte forbundet med den kvalitative undersøgelse og egner sig især til at undersøge nye emner før man har viden nok til at fremsætte en hypotese. 
\\
Grundet manglen på en hypotese forud for undersøgelsen, anvendes der således ikke en deduktiv metode i dette projekt. Deduktion anvendes oftest til at undersøge en hypotese der er opstillet på forhånd, og ikke til at opbygge den. Man kunne eventuelt fortsætte sin induktive undersøgelse med at undersøge sin nye hypotese deduktivt. 
\\ \\
ret ting her liv!!!


\section{Interview opbygning}
Der skal stå noget her!!! \\
\\



\subsection {Struktur af interviewet}
I denne undersøgelse er det valgt at foretage interviewene på en semistruktureret måde. Her forberedes og gennemgås fastlagte spørgsmål,  men den fleksibilitet der kendertegner en almindelig samtale bibeholdes. Dette giver intervieweren mulighed for at gå igennem de emner, de har behov for at lære mere om, og samtidig kan der stilles uddybende spørgsmål, eller interviewpersonen kan påvirke samtalen med deres eget perspektiv \citep{brinkmann2014}.\\
Dette gøres, fordi formålet med undersøgelsen er at finde et problem der kan løses. Der er hermed mulighed for at kunne spørge ind til de emner, som kunne lede til identificeringen af et sådant problem. Ligeledes giver den semistrukturerede metode interviewpersonen mulighed for selv at bringe emner på banen, hvis de tror de kan være relevante.

\subsection {Medie for interviewet}
Det er besluttet at lave interviewene ansigt til ansigt, da denne måde at interviewe på giver adgang til informationer gennem kropssprog, i modsætning til fx interviews over telefonen eller internettet. Svend Brinkmann anbefaler derfor at intervieweren selv transskriberer interviewet.\citep{brinkmann2014} I denne undersøgelse antager vi dog, at dette ikke er strengt nødvendigt, da de emner, som interviewet skal belyse, ikke er personlige for den interviewede, og således er deres kropssprog antaget til at være mindre relevant.
\\
\\
\\
\\
Hvordan interviewer vi folk? Forskningsspørgsmål og interview spørgsmål.
\\
\\
Motiver. Brinkmann side 39.
\\
\\
Ifølge Steiner Kvale skal man interviewe det antal personer der er nødvendigt, for at opnå resultater. Derudover angiver han at antallet almindeligvis er 15 +/- 10, så vi er ikke helt hvor vi skal være, men det er også okay da det er en undersøgelse med lille omfang.
\cite{kvale2015}
\\
\\
Hvordan transkriberer vi? 
\\
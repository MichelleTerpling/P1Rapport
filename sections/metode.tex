\chapter{Metode}\label{ch:ch2label}
Vi har foretaget en undersøgelse med interviews, med det formål at finde potentielle problemer indenfor ressourcefordeling i idrætsforeninger. Valget at undersøgelsesmetode er vigtigt fordi forskellige metoder vil giver forskellige resultater. I dette kapitel vil der blive argumenteret for valg af undersøgelsesmetoder.


\section{Valg af metode}
% Talesprog, i know.....
Vores undersøgelse benytter sig af en kvalitativ induktiv fremgangsmåde. Vi ved ikke nok om hvordan idrætsforeninger fungerer ved projektets start, så vi har derfor brug for at skabe baggrundsviden. Dette ville ikke kunne fungere med en kvantitativ metode, da vi endnu ikke ved hvad der skal stå i spørgeskemaet, og derfor ikke ville kunne få meningsfyldt data tilbage. I stedet vil vi interviewe mennesker, der arbejder i de forskellige idrætsforeninger.\\ 
Ved at interviewe informanterne, kan vi stille større spørgsmål som ikke bare giver svar som "ja", "nej", "enig"\ eller "uenig". Der er også mulighed for at følge op på spørgsmålene med det samme, og få svarene uddybet. Dette er ikke en mulighed med en kvantitativ fremgangsmåde, da det foregår igennem spørgeskemaer, hvor kommunikation mellem interviewer og informant, ikke kan ske på en praktisk måde, ud over den første mail.\\
\\
Der kunne argumenteres for at vi skulle bruge kvantitativ metode, efter vi har gennemført de forskellige interviews. På denne måde kunne vi se om de problemer interviewene afdækker, er generelle, eller kun gælder i den lokale idrætsforening. Vi mener dog at dette ikke er nødvendigt for vores undersøgelse, da det er en pilotundersøgelse, som bare skal se om der overhovedet er problemer vi kan løse.\\


% Projektet handler om ressourcefordeling inden for idrætsforeninger. For at finde det gode problem har vi derfor valgt at lave interviews med forskellige idrætsforeninger med forskellige sportsgrene. På denne måde kommer vi ud og ser de rigtige problemer i virkeligheden, på en måde som et spørgeskema (kvantitativ metode) ikke kan. Hvis man skal lave spørgeskemaer, skal man have en bedre ide om hvad man vil have svar på, og dette har vi endnu ikke i vores projekt. \\

\section{Interview}
% Her skal vi nok også skrive noget klogt...

























% Her samles vi 
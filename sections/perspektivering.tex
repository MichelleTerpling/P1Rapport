\chapter{Perspektivering}\label{ch:perspektivering}
% Introduktion: en vurdering af om vores program er en god løsning eller ej.
I dette kapitel vil der blive undersøgt, om programmet er en god løsningen i forhold til eksisterende programmer, der sammensætter stævneplaner, og andre sportsgrene. Der vil blive diskuteret alternative metoder at lave løsningen på. Til sidst vil der blive diskuteret, hvilke slags udvidelser der potentielt kunne laves til programmet.

\section{Eksisterende værktøjer}
Der findes allerede værktøjer, som sammensætter stævneplaner. Eksempler på disse værktøjer er Tournament Scheduler og Teamopolis, som også er blevet analyseret og vurderet i afsnit \ref{eksisterendeProgrammer} (s. \pageref{eksisterendeProgrammer}) i forbindelse med deres effektivitet og fleksibilitet. For at vurdere hvor godt projektets program er, vil det i dette afsnit blive sammenlignet med disse eksisterende værktøjer.
\\\\
Projektets program er lavet, så den følger Kidzliga floorball reglerne. De eksisterende værktøjer, fra afsnit \ref{eksisterendeProgrammer}, er dog ikke begrænset til denne form for stævne, og er derfor allerede anderledes på det punkt. Både Tournament Scheduler og Teamopolis er stævneplanlæggere, men de er begrænset i form af, at de ikke rigtig er inddelt i runder baseret på antal af baner, der er til rådighed. I stedet antages det, at alle hold spiller i hver runde. 
\par
I modsætning til dette, tager projektets program højde for denne begrænsning, og laver runder afhængigt af antallet af baner, som brugeren har til rådighed. Derudover giver denne egenskab også mulighed, for at et hold kan holde pauser mellem hver kamp, hvor de andre to værktøjer ikke tager højde for dette. Denne egenskab fungerer dog ikke altid, da der er tilfælde, hvor der er enkelte hold, som er nødt til at spille op til to gange i træk med den hurtige metode for generering af stævneplaner. Det er mindre sandsynligt med den bedre metode, men det kan dog stadigvæk ske. Det vurderes til at være bedre, end at holdet skal spille i hver runde. 
\\\\
Med de eksisterende planlægnings-værktøjer er brugeren nødt til at lave en hel ny stævneplan, hvis der ønskes at tilføje eller fjerne et hold. Med projektets program kan den tidligere stævneplan indlæses, hvorefter man indtaster de ønskede ændringer. Denne måde gør det nemmere og hurtigere for brugeren at ændre på planen, da de ikke skal indtaste alle holdnavnene igen.
\par
Projektets program har heller ikke det problem, at hver gang en stævneplan lægges, vil den ligne den forrige plan, da holdene sammensættes tilfældigt i kampe. Med de eksisterende værktøjer indsættes hvert hold i rækkefølge efter listen af holdnavne, og derfor er stævneplanen afhængig af rækkefølgen af navnene. Hvis der ønskes en stævneplan, som er forskellig fra den tidligere plan, er man nødt til manuelt at ændre rækkefølgen af holdnavene, som de opstår i listen.
\par
Et andet problem med disse eksisterende værktøjer er, at der ingen begrænsning er, for hvor mange kampe et hold skal spille. I stedet er stævneplanen sammensat, så alle hold spiller mod alle. Dette er dog ikke optimalt, hvis der eksempelvis er for mange hold. Derudover tages der ikke højde for, hvilket niveau hver hold har. I modsætning til dette, tager projektets program højde for disse ting. De krav, der er sat for programmet, sørger for, at et hold ikke spiller for mange gange, og at de ikke spiller mod et hold, der har for højt eller lavt niveau.
\\\\
Disse eksisterende planlægnings værktøjer kan derfor ikke bruges til at lave kidzliga stævner i floorball, da de ikke følger reglerne, som stævnet følger. De nuværende planlægnings-værktøjer har heller ikke den sammen fleksibilitet, som projektets program har, og som projektets problemformulering handlede om at forbedre. Det kan derfor konkluderes, at projektets program er en bedre stævneplanlægger for kidzliga stævner i floorball, end de eksisterende planlægnings værtøjer er.  

\section{Alternative løsninger}
Udover den måde programmet er blevet udformet på i projektet, er der også andre metoder til sammensætning af stævneplaner.
\par
Som der blev nævnt i problemløsnings-kapitlet (se kapitel \ref{ch:problemformulering}), bliver stævneplanen sat sammen tilfældigt. Dette er dog én metode ud af mange programmet kunne benytte.
\\\\
I stedet for den tilfældige metode kunne man lave en algoritme, som altid vil sammensætte en stævneplan, der opfylder alle de stillede krav. Denne metode bliver brugt i de eksisterende værktøjer fra afsnit \ref{eksisterendeProgrammer} (s. \pageref{eksisterendeProgrammer}) til at oprette en stævneplan. Der bruges ikke lang tid på at generere en stævneplan, hvilket er fordelagtigt. Med denne metode vil der altid sammensættes en god stævneplan som overholder alle krav. Der vil dog være en risiko, for at den bedste plan ikke vil blive fundet. Dette skyldes, at der gennemgås et begrænset antal sammensætninger af stævneplanen, og det kan ikke garanteres at den bedste plan er iblandt dem.
\par
En anden metode ville være at få programmet til at generere samtlige permutationer af en stævneplan. Disse planer ville også skulle evalueres i forhold til de krav, der er stillet til programmet. Hermed er der ingen risiko for at overse den bedste plan, da den altid vil blive produceret. Denne metode er dog ikke den mest optimale, da det er mange forskellige stævneplaner, som skal sættes sammen.
\\
Dette tager for lang tid for programmet at gøre, da det løbende skal evaluere hver version af stævneplanen. For at skære ned i antallet af stævneplaner, der skal sammensættes, er der i stedet blevet valgt en metode, hvor hold sammensættes tilfældigt i kampe, og derefter sættes de ind i runder, der evalueres så den mest optimale stævneplan vælges. Med denne metode er der dog stadig en risiko, for at den bedste stævneplan ikke bliver fundet. På den anden side er denne metode langt hurtigere end at sammensætte alle mulige stævneplaner.
\\\\
Løsningen til problemformuleringen blev taklet ved at lave et program, der genererer stævneplaner tilfældigt og evaluere dem. Denne metode vurderes til at være passende til at løse problemet, men der er andre metoder og tilgange programmet kunne skabes ud fra, der alle har forskellige fordele og ulemper.

\section{Udvidelse af programmet}
Hvis der havde været flere ressourcer til rådighed, ville det være muligt at videreudvikle programmet. 
\\\\
Den stævneplan, der bliver genereret af programmet, er meget stringent i forhold til tiden. Der er sat præcis otte minutter af til hver runde, og bliver denne overskredet, falder resten af stævneplanen fra hinanden. Derfor kunne det være en god ide at tilføje en indbygget tidstager, så brugeren ikke selv skal koordinere, hvor lang tid der går mellem hver runde. 
\par
Dette kunne gøres ved at lave en funktion, der scanner stævneplanen igennem, og finder de tidspunkter, hvor hver runde starter. Derefter kunne en anden funktion vise, hvor lang tid der er tilbage af den nuværende runde, og så lave en lyd når kampen er ovre. Herefter kunne den lave en anden lyd, når pausen er forbi, og den næste runde skal starte. På denne måde forbliver hele stævnet forhåbentligt synkroniseret, og der er ikke nogen, som går i gang for sent eller for tidligt.
%\par
%Denne funktionalitet er dog ikke blevet implementeret, da det blev set som en ekstra ting, programmet ikke har brug for at gøre for at kunne fungere.
\par
Videreudvikling kunne også være at gøre programmet bredere, så det også kan lave turneringsplaner egnet til andet end Kidzliga floorball.
\\\\
Da programmet specifikt følger de regler og krav, der er stillet af Floorball Danmark i forbindelse med Kidzliga stævner, er det meget begrænset til dette område. Man kunne dog gøre programmet bredere, så det ikke kun er afgrænset til Kidzliga reglerne, men så der også er mulighed for at få planer til eksempelvis pokalturneringer. Der vil hermed være en valgmulighed, hvor man kan vælge mellem hvilken form for turnering, der skal dannes en plan for. Tilhørende input tastes derefter ind af brugeren, og en passende plan vil blive dannet.
\par
Da der er andre regler og en anden opsætning i forhold til pokalturneringer, vil programmet også se anderledes ud. Pokalturneringer er sat op, så man kun skal sammensætte hold i kampe én gang. De næste kampe afhænger af hvilket hold, der har vundet i den forrige kamp. Der vil derfor være behov for at udvikle en ny del til programmet, som sørger for at fremstille denne form for turneringsplan. 
\\\\
Programmet kan udvides, så det også kan lave planer til andre sportsgrene end floorball. Hvis man gør det muligt for brugeren at ændre på flere variable. Disse variable kunne eksempelvis være i forbindelse med tiden, hvor man kunne gøre varigheden af hver kamp, pauserne mellem hver kamp og eventuelle andre pauser, til variable brugeren kan bestemme.
\par
Alternativt, kunne man også lave skabeloner, som passer til de forskellige sportsgrene, hvor man har mulighed for at vælge hvilken sport, man vil lave en plan til. Med disse skabeloner vil brugeren ikke være nødt til at taste ting ind på ny, men der er nogle allerede indtastede informationer, som der kan ændres på hvis nødvendigt gennem standard input, \textbf{\textit{stdin}}. 
\\\\
Disse udvidelser vil kunne gøre programmet anvendelig for en bredere målgruppe og derfor gøre programmet mere fleksibelt og brugbart for forskellige idrætsforeninger, mens programmet beholder dets fleksible tilgang.

\section*{Opsamling}
I dette kapitel er programmet blevet sammenlignet med eksisterende værktøjer, alternative løsninger og mulige udvidelser af programmet. På denne måde er programmet sat op imod noget mere end bare problemformuleringen, for at udvide perspektivet. I forhold til omfanget for dette projekt, er programmet et godt udgangspunkt, men det kan forbedres med flere ressourcer. 


% HecK YeAh
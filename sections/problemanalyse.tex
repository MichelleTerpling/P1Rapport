\chapter{Problemanalyse}\label{ch:ch2label}


% \section{Hvad skal vi finde ud af og have med i rapporten?}
% Hvordan fungerer en sportsklub?\\
% Hvilke ressourcer skal de holde øje med?\\
% Hvad gør de nu, og hvordan fungerer det?\\
% Er der et problem, og hvordan er det et problem?\\
% Hvad gør en større klub?\\
% Hvad er der specielt ved en mindre klub i forhold til en større?\\
% Hvem er interessenterne?\\

% \section{Hvad gør vi?}
% Hvem interviewer vi?\\
% Og hvem interviewer dem? Os alle sammen...\\

Det initierende problem for dette projekt er ressourcefordeling, og specielt organisering af disse ressourcer. 
\\ 
\\
HER SKAL DER STÅ HVAD FOKUS ER MED PROBLEMANALYSE.

\subsection*{Hvordan hænger en forening sammen?}
%Afsnit flyttet til indledning




% % % % % % % % % % % % % % % % % % % % % % % % % 
\section{Interessentanalyse}
\textbf{Dette afsnit skal laves om efter vi har fundet problemet vi vil afgrænse os til, så det giver mere mening}
\\ \\
Interessenterne er en stor del af projektet, da der skal laves et program specifikt for at løse deres problem. Denne undersøgelse vil blive brugt til at finde en retning for projektet og et mere afgrænset problem, som kan løses. Det er hermed relevant at forstå hvem interessenterne er og hvad deres behov i forhold til problemet og dets løsning er. Derudover er det også vigtigt at forstå de forskellige interessenters intentioner og motivation for projektet.

\subsection{Bestyrelse}
En af de større interessenter er bestyrelsen. Medlemmerne inden for bestyrelsen er som regel frivillige. Som bestyrelse har de stor indflydelse på projektet, da det er dem løsningen skal hjælpe. De kommer på den måde til at blive påvirket af projektet. Bestyrelsen har brug for en bedre og hurtigere måde at planlægge og organisere deres udstyr og tid på. Ulempen ved dette er at, hvis ikke alle bruger systemet, kan der være sammenstød eller uoverensstemmelser med planlægningen. Det vil hermed være svært at holde styr på systemerne, hvis folk ikke bruger det samme program eller planlægningsmetode.\\ 
Bestyrelsen vil være i stand til at bidrage med hvilke features der er nødvendig i forhold til løsningen. De vil hermed være med til at afklare hvor problemet helt specifikt ligger, og hvilke krav de har til løsningen. Det er hermed nødvendigt at kontakte bestyrelsen, for at kunne høre deres holdninger i forhold til løsningen, samt hvorvidt de har problemet eller ej. Dette vil ske gennem et semistruktureret interview, hvor interessenten får lov til at dele sine holdninger frit.

\subsection{Frivillige}
De frivillige er også en vigtig interessent, fordi den anvendte planlægningsmetode i klubben har en væsentlig indflydelse på deres arbejdsopgaver. Da det er dem der skal bruge programmet, er det i deres interesse at planlægningen bliver optimeret, så der skal bruges mindre tid på planlægningen. Deres behov kan hermed påvirke programmets udformning. Ulempen ved at introducere en ny metode, er at de frivillige skal lære at bruge programmet. Fordelen ved programmet er dog at det bliver nemmere og mindre tidskrævende for den enkelte frivillige at planlægge og holde styr på de forskellige områder. \\ Det er hermed nødvendigt at kontakte denne interessent, for at få afklaret hvilke behov der er i forhold til programmet. Dette vil igen gøres gennem et semistruktureret interview.

\subsection{Medlemmer}
Medlemmerne har en mindre rolle i forhold til hvordan løsningen udvikles. De har ikke nogen indflydelse på programmet, da det ikke vil være dem der bruger det. De vil dog blive påvirket af programmet, da optimering af klubbens ressourcer vil have en indflydelse på deres aktiviteter. Hvis programmet ikke bliver brugt ordentligt eller ikke er effektivt nok, vil det få negative konsekvenser for medlemmerne, da der også er tale om deres tid. Medlemmernes holdninger og ønsker i forhold til løsningen af problemet vil ikke blive taget højde for, da løsningen er rettet mod de frivillige og bestyrelsen.

%\subsection{Forbundet}
% Kun relevant i forhold til turneringer

\subsection{Forældrene}
Forældrene, hvis børn er et medlem af foreningen, er interessenter med en minimal indflydelse på programmets udvikling. De kommer ikke til at bruge programmet, de vil dog kunne mærke programmets virkning, især hvis der er sket fejl i planlægningen, og det så går ud over forældrenes tid og lyst til at deltage i aktiviteter. Hvis der eksempelvis ikke er styr på faciliteter og der så sker dobbeltbookning af halen og forældrene har kørt børnene til aktiviteten. På den måde er forældrenes tid blevet spildt. Dette vil kunne undgås hvis man har et fungerende og effektivt program, der kan holde styr på disse ting.
\\
% Det er her vi samles

\section{Slitage}

\chapter{Problemformulering}\label{ch:chlabel}
Gennem interviews med forskellige idrætforeninger er det blevet klart, at der ikke altid findes optimale planlægningsværktøjer til organiseringen af foreningernes ressourcer. Ofte er dette ikke et problem for foreningerne, men i forbindelse med planlægningen af turneringer vil et program kunne hjælpe foreningerne med en hurtigere og mere fejlfri programlægning. \\
Det er blevet valgt at fokusere på stævner, da større turneringer ikke planlægges af den enkelte forening. KidzLiga, indenfor floorball, bruger stævne-konceptet, og vil danne rammen for resten af dette projekt. Der vil være fokus på at nedsætte tiden, der bruges til at lægge et kampprogram for turneringen, samt at sikre fleksibilitet ved at give mulighed for at lave rettelser. Dette er en vigtig funktion, da afbud ofte kræver store ændringer i kampprogrammet, hvilket kan være tidskrævende.
\\\\
Med dette formål i sigte stilles følgende problemformulering:
\paragraph{Hvordan kan man, med en softwareløsning, gøre det enklere for turneringsplanlæggere at oprette og redigere KidzLiga turneringsplaner?}


\chapter{Problemformulering}\label{ch:chlabel}
Gennem interviews med forskellige idrætforeninger er det blevet klart, at der ikke altid findes optimale planlægningsværktøjer til organiseringen af foreningernes ressourcer. Ofte er dette ikke et problem for foreningerne, men i forbindelse med planlægningen af turneringer vil et program kunne hjælpe foreningerne med en hurtigere og mere fejlfri programlægning.
\par
Det er blevet valgt at fokusere på stævner, da større turneringer ikke planlægges af den enkelte forening. KidzLiga, indenfor floorball, bruger stævne-konceptet, og vil danne rammen for resten af dette projekt. Der vil være fokus på at nedsætte tiden, der bruges til at lægge et kampprogram for turneringen, samt at sikre fleksibilitet ved at give mulighed for at lave rettelser. Dette er en vigtig funktion, da afbud ofte kræver store ændringer i kampprogrammet, hvilket kan være tidskrævende.
\\\\
Med dette formål i sigte stilles følgende problemformulering:
\par
\textbf{Stævneplanlægning til Kidzliga i floorball er tidskrævende, og der kan opstå fejl, når det gøres manuelt. Derudover er det svært at lave hurtige ændringer i stævneplanen.}
\\\\
Med udgangspunkt i problemformuleringen vælges det at løse problemet med et computerprogram. Et program vil kunne oprette og redigere stævneplaner væsentligt hurtigere, og med færre fejl, end det kan gøres manuelt. Der vil således også være mulighed for, at redigere en stævneplan i sidste øjeblik før stævnet starter uden stort besvær, og samtidig undgå at fejl opstår i stævneplanen.
\par
Programmet vil blive skrevet i programmeringssproget C, da det er det sprog projektgruppen har kendskab til. Det er derudover et krav for projektet at skrive programmet i dette sprog.


%\subsubsection{Hvordan kan man gøre det enklere for turneringsplanlæggere at oprette og redigere KidzLiga turneringsplaner?}

%\subsubsection{Det er svært at oprette og redigere i turneringsplaner der lægges for kidzliga indenfor floorball.} % Det skal omformuleres så der er mere fokus på fleksibilitet.



















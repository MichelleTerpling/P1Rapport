\chapter{Problemløsning}\label{ch:chlabel}

% Hvad skal man tage højde for?\\
 % - Regler\\
 % Der er et link på Discord om reglerne i Kids floorball\\
   % - Yngre og ældre spillere\\
   % - Kids koncept\\
     % - Kampe tager 6 minutter "løbende tid"
     % - 1-2 minutters pause mellem kampe
     % - Skal have en hvilekamp mellem hver kamp, hvis ikke muligt skal kampene gerne foregå på samme bane.
     % - Alle kampe skal starte og slutte på samme tid
     % - Der er forskellige niveauer, indeles efter skills og ikke alder.
 % - Afmeldinger\\
 % Der opstår problemer i planlægningen, når folk melder fra\\
 % - Andre ting\\
 % floorball.dk/staevner/

\begin{minted}[frame=lines, framesep=3mm, baselinestretch=1, linenos, bgcolor=LightGray]{c}
/* Dette er et program der printer "Hello World" 10 gange */

# include <stdio.h>

int main(void) {
    
    printf("Hello World!");
    
    for (int i = 1; i <= 10; i++) {
        printf("Hello World, %d!\n", i);
    }
    
    return 0;
}
\end{minted}
\\
Input:\\
 - (unsigned int) Klokkeslæt\\
 - (int) baner\\
 - (int) hold\\
 - (unsigned int) Tid per kamp\\
 - (unsigned int) Tid mellem kampe\\ 
 \\
 Output: \\
  - En .csv fil (kan åbnes i Excel), med turneringen\\
\\
Funktioner: \\
 - Skal tage højde for at hold ikke skal spille mange gange i træk.\\
 - Skal kunne opdateres, så hvis nogen ikke dukker op, kan planen laves igen.\\
 - Skal kunne tage resultatet af kampe ind, og skrive ud hvem der skal spille mod hinanden nu.\\
 
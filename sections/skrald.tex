Der er flere divisioner på forskellige niveauer. Kun de øverste divisioner har opryknings- og nedrykningskampe, hvilket gør, at der ikke er for stor forskel på niveauet mellem holdene i hver division. \cite{Landsturnering}. Hold, der ligger øverst i den næstøverste division, har mulighed for at kvalificere sig til at spille i den øverste liga, hvis de formår at vinde oprykningskampen. I oprykningskampen skal de spille imod de hold der er nederst i den øverste division. Vinderne af kampene rykker op, og taberne rykker ned.\\



\subsection*{Cup-systemet}
Der findes flere forskellige løsninger til turneringsplanlægning ud fra cup-systemet. I dette projekt er der blevet fundet to sådanne løsninger, som vil blive analyseret her. Disse to hedder Cup Fixture Generator (CFG) og Challonge \citep{CupFixtureGenerator}\citep{challonge}.\\
De genererer begge et kampprogram ud fra en liste af hold, men gør det på lidt forskellige måder. CFG laver et kampprogram der er meget uoverskueligt og kræver at brugeren husker hvem der vinder hver kamp. Challonge genererer et mere intuitivt kampprogram, hvor brugeren nemt kan overskue turneringens forløb.\\
De har dog den begrænsning, at de ikke kan lave andet end turneringer med cup-system, hvilket ikke er brugbart i forhold til KidzLiga.

Cup Fixture Generator er en hjemmeside, hvor man får en liste med hvem der spiller i hver kamp. Listen følger cup-systemet, og er delt op i forskellige runder afhængig af hvor mange hold der spiller. Det eneste input der kræves af brugeren er antallet af hold der deltager og holdenes navne. Hjemmesiden har kun den enkelte funktion; at printe en liste ud med turneringsplanen. Måden hvorpå det er opstillet er ikke optimal. Der er ikke mulighed for at lave ændringer eller indtaste hvem vinderen er, og hermed hvem der skal videre i næste runde. I stedet står der under den næste runde, at vinderen af eksempelvis den første og anden kamp skal spille mod hinanden. Man mister på denne måde nemt overblikket, da det bliver nødvendigt at gå tilbage til tidligere kampe, for at se hvem der vandt og derefter vide hvem der skal spille i den næste kamp. Dette program er derfor begrænset da man kun kan producere én type turneringsplan, hvilket ikke er optimalt i forbindelse med KidzLiga, da man ikke ønsker at have fokus på konkurrenceelementet .

Challonge er en hjemmeside, hvor man kun indtaster holdnavnene. Ligesom Cup Fixture Generator får man som output en turneringsplan baseret på cup-systemet. Denne hjemmeside er dog mere fleksibel, da der er mulighed for at indtaste hvem der har vundet i hver runde. Når dette er registreret, bliver holdnavnet skrevet ind i den næste kamp hvis de har vundet. Dette er med til at skabe overblik over hvilke hold der har vundet, og det er ikke op til brugeren at skulle hold styr på hvem der skal spille i næste runde, da den selv opdaterer planen. Challonge har dog den samme begrænsning som Cup Fixture Generator har. Der er kun mulighed for at lave en plan baseret på cup-systemet \citep{challonge}. 

Hjemmesiden Tournament Scheduler genererer turneringsplaner baseret på det system hvor alle hold mødes i en kamp mod hinanden gennem turneringen. Når man kommer ind på hjemmesiden bliver man bedt om at indtaste forskellige oplysninger som holdnavne, lokation og om holdene skal spille mod hinanden en eller to gange. Derefter får man en liste over hvilke holde, der spiller mod hinanden i hver runde. Der er mulighed for at redigere listen, hvor man kan ændre holdnavne samt tilføje point og eventuelle noter. Hjemmesiden generer også en rankingside, hvor man kan få et overblik over hvilke hold der eksempelvis har vundet eller tabt flest runder. Der bliver også udregnet hvor mange gange hver hold har spillet på det nuværende tidspunkt i turneringen.\\
Begge sider har dog visse begrænsninger. Med Tournament Scheduler er der kun mulighed for at lave en turneringsplan, hvor alle hold har spillet mod hinanden. Dette kan være en ulempe, hvis der er mange hold, som skal spille, da det kan tage for lang tid at komme i gennem alle kampene \citep{TournamentScheduler}.
Der er heller ikke mulighed for at randomize holdene. Så skal man selv skrive navnene ind i en anden rækkefølge

\subsection*{Teamopolis}
Teamopolis minder meget om Tournament Scheduler. Teamopolis laver også turneringsplaner baseret på at alle hold spiller mod hinanden gennem turneringen. Man bliver bedt om de samme oplysninger som fra Tournament Scheduler, hvorefter turneringsplanen er givet. Her kan man se hvem der spiller mod hvem i hver runde, og hvorhenne. Derudover er der mulighed for at udvælge en bestemt del af turneringsplanen, som man vil se på. Man kan eksempelvis kigge på et bestemt hold, og se hvilke hold det skal spille mod i hvilke runde.\\
Der er dog ikke muligheden for at få et overblik over hvilke hold har flest point, og sammenligne sig selv med andre, som det var muligt med Tournament Scheduler. Derudover er det heller ikke muligt at blande holdene, i tilfældet at man ønsker en anden holdopsætning. Brugeren er selv nødt til at ændre på rækkefølgen af holdnavnene. Hjemmesiden giver, ligesom Tournament Scheduler, kun mulighed for at lave turneringsplaner, hvor alle spiller mod hinanden, og har derfor samme ulemper \citep{Teamopolis}.




\section*{Stikord - Krav til program}
\begin{itemize}
    \item KidzLiga regler
    \begin{enumerate}
        \item Kampene varer 6 minutter løbende tid.
        \item 1 til 2 minutter mellem hver kamp.
        \item Alle hold skal helst have omkring 6 kampe (+/- 1).
        \item Alle kampe skal starte og slutte på samme tid. 
        \item Et hold skal helst ikke spille to kampe i træk, de skal helst have én hvile kamp efter hver kamp som minimum. Hvis dette ikke er muligt, skal den næste kamp spilles på den samme bane.
        \item Der er 4 niveauer i stigende alder og erfaring: N, A, B og C.
        \item Kampprogrammet skal hænges op i hallen, og skal derfor være præsentabel.
        \item Det forventes at stævnet varer maks. timer.
    \end{enumerate}
    
    \item Input:
    \begin{enumerate}
        \item Holdnavne i en liste, med niveau. Skal stilles op på en bestemt måde.
        \item Antal baner.
        \item Starttidspunkt for stævnet.
        \item Eventuelle pauser.
        \item En færdig plan der skal opdateres.
        \item Hold der skal fjernes eller tilføjes.
    \end{enumerate}
    
    \item Output:
    \begin{enumerate}
        \item Skal lave kampprogram.
        \item Skal vise rundenummer.
        \item Startidspunkt for hver runde.
        \item Hvilke hold der spiller mod hvem.
        \item To hold må helst ikke spille mod hinanden 2 gange i træk. Hvis der er hold nok, må to hold ikke spille mod hinanden flere gange.
        \item Hvilke hold der holder pause.
    \end{enumerate}
    
    \item Funktioner:
    \begin{enumerate}
        \item Funktion der kan fjerne og tilføje hold, baseret på runder, og sørger for at tidligere runder ikke bliver lavet om. 
    \end{enumerate}
    
\end{itemize}

% floorball.dk/staevner/


Eksempel på snippet (Ikke fra et egentlig program):
\begin{minted}[frame=lines, framesep=3mm, baselinestretch=1, linenos, bgcolor=LightGray]{c}
/* Dette er et program der printer "Hello World" 10 gange */

# include <stdio.h>

int main(void) {
    
  for (int i = 0; i < 10; i++) {
    printf("Hello World, %d!\n", i);
  }
     
  return 0;
} 
\end{minted}

 
\begin{minted}[frame=lines, framesep=3mm, baselinestretch=1, linenos, bgcolor=LightGray]{c}

typedef struct hold {
    char holdnavn[60];
    int niveau;
    /* int antal kampe spillet */
} hold;

typedef struct kamp {
    char hold_a[60];
    char hold_b[60];
    int niveau;
} kamp;

int main(void) {

    int stævne[antal runder];
    
    struct kamp runder[antal baner];

    return 0;
}
\end{minted}

%\begin{enumerate}
%    \item 1 til 2 minutter mellem hver kamp.
 %   \item Alle hold skal helst have omkring 6 kampe (+/- 1).
 %   \item Alle kampe skal starte og slutte på samme tid. 
 %   \item Et hold skal helst ikke spille to kampe i træk, de skal helst have én hvile kamp efter hver kamp som minimum. Hvis dette ikke er muligt, skal den næste kamp spilles på den samme bane.
 %   \item Der er 4 niveauer i stigende alder og erfaring: N, A, B og C.
%    \item Kampprogrammet skal hænges op i hallen, og skal derfor være præsentabel.
 %   \item Det forventes at stævnet varer maks. 6 timer.
%\end{enumerate}
% En lille eksempeltekst, der viser hvordan kravene alternativt kan opstilles.  
% Krav på baggrund af regler

% Ikke alle af disse regler har dog indflydelse på hvordan et kampprogram skal stilles op. 
% Da der tages udgangspunkt i kidzliga-stævner, er det oplagt, at de tilhørende regler har indflydelse på, hvordan et kampprogram for et stævne skal stilles op.

% Der er også krav til løsningen, der har til formål at gøre det lettere for en bruger at anvende programmet. Det indebærer blandt andet krav til yderligere indhold.
% Kravene til det resulterende kampprogram kan ses i listen nedenfor.

 %   \begin{enumerate}
  %      \item Skal lave kampprogram.
   %     \item Skal vise rundenummer.
%        \item Startidspunkt for hver runde.
%        \item Hvilke hold der spiller mod hvem.
%        \item To hold må helst ikke spille mod hinanden 2 gange i træk. Hvis der er hold nok, må to hold ikke spille mod hinanden flere gange.
%%   \end{enumerate}
%
%For at en løsning skal være i stand til at overholde disse krav, har det brug for visse informationer fra brugeren. Kravende til brugerinput kan ses i den nedenstående liste.
%    \begin{enumerate}
%        \item Holdnavne i en liste, med niveau. Skal stilles op på en bestemt måde.
 %       \item Antal baner.
  %      \item Starttidspunkt for stævnet.
  %      \item Eventuelle pauser.
  %      \item En færdig plan der skal opdateres.
  %      \item Hold der skal fjernes eller tilføjes.
   % \end{enumerate}


% \\\\
% Programmet blev til at starte med, udviklet med den tanke, at der skulle være to forskellige funktioner der redigerede stævneplanen. Disse blev navngivet \textbf{\textit{addTeams}} og \textbf{\textit{removeTeams}} (Se bilag \ref{ch:appFlabel}). De blev kombineret til \textbf{\textit{modifyTeams}}. Funktionerne tog sig af hver deres del af redigeringen, og gjorde det på lidt forskellige måder. I \textbf{\textit{addTeams}} bliver \textbf{\textit{number\_of\_teams}} talt op \textit{før} der bliver allokeret plads til \textbf{\textit{all\_teams}} arrayet. Dette er fordi der skal være plads til de nye hold. I \textbf{\textit{removeTeams}}, er dette ikke nødvendigt, da de hold der bliver fjernet, får markeringen \textbf{\textit{EMPTY}}, der gør at resten af programmet ignorerer dette hold.
% \par
% Den måde hver funktion opfører sig på, alt efter om det er første gang de bliver kaldt eller ej, er også lidt forskellig. I begge funktioner bliver holdnavne og niveau scannet ind den første gang, men efter første gang, udvides \textbf{\textit{all\_teams}} i \textbf{\textit{addTeams}}, men dette sker ikke i \textbf{\textit{removeTeams}}. Det skyldes som sagt at holdene ikke bliver fjernet, men bare bliver markeret med \textbf{\textit{EMPTY}}.
% \par
% Der opstår også en lille forskel mellem funktionerne, når de nuværende hold skal printes med \textbf{\textit{printTeams}}. I \textbf{\textit{addTeams}}, skal antallet af nye hold trækkes fra \textbf{\textit{number\_of\_teams}}, da print funktionen ellers ville gå ud over den plads der er allokeret til \textbf{\textit{all\_teams}}. Igen er dette ikke nødvendigt i \textbf{\textit{removeTeams}}, da \textbf{\textit{number\_of\_teams}} ikke har ændret sig.
% \par
% I forhold til funktionen \textbf{\textit{getTeams}}, er der forskel på nogle af inputparametrene, da der er forskel på den måde informationer skal scannes ind fra terminalen. Funktionen \textbf{\textit{addTeams}}, bliver nødt til at kende både holdets niveau og navn, \textbf{\textit{removeTeams}} behøver kun holdets navn.
% \par
% Endeligt er der forskel på de hvordan hver funktion afsluttes. Funktionen \textbf{\textit{addTeams}}, afsluttes med at kalde funktionen \textbf{\textit{copyTeams}}, som kopierer alle de tilføjede hold over i \textbf{\textit{all\_teams}}. I \textbf{\textit{removeTeams}} bliver de hold der skal fjernes, markeret med \textbf{\textit{EMPTY}} ved brug af funktionen \textbf{\textit{deleteTeams}}, hvorefter disse hold bliver flyttet til de sidste pladser i \textbf{\textit{all\_teams}} med funktionen \textbf{\textit{sortArrayByLevel}}.


% Denne test kan ikke sammenlignes direkte med, hvor hurtigt et menneske kan lægge en stævneplan, da dette ikke er testet.  
%Det antages, at programmet er i stand til at lave en stævneplan hurtigere end et menneske. 
% På trods af at der ikke er foretaget nogen test i dette projekt, der afgør, hvor hurtigt et menneske kan gøre det. Men der er blevet foretaget tests af, hvor lang tid visse funktioner i programmet tager. 


% \section{Test}
% Test af funktioner

%\subsubsection{Fjern et hold}
%I dette afsnit gennemgås \textbf{\textit{removeTeams}}-funktionen, med forklaringer og argumenter for den valgte implementering. Funktionens overordnede funktion er at fjerne de hold brugeren, skriver som input, fra et eksisterende kampprogram.
%\par
%Denne funktion bliver kaldt i \textbf{\textit{editMenu}}-funktion, hvis brugren vælger at slette et eksisterende hold. \textbf{\textit{removeTeams}}-funktionen kaldes med en fil-pointer til "turneringsplan.txt", en variable kaldet sential bliver også kaldt. \textbf{\textit{all\_teams}}-arrayet bliver også kaldt som en pointer og med en int pointer til \textbf{\textit{number\_of\_teams}}. Funktionen returnere et \textbf{\textit{all\_teams}}-array af typen team, som bliver brugt videre i processen i at lave et redigeret kampprogram. 
%\par
%Når funktionen køres bliver variablerne int \textbf{\textit{team\_index}}, int \textbf{\textit{number\_of\_removed\_teams}} og team \textbf{\textit{*removed\_teams}}, som er en pointer til et array, initialiseret.
%I linje ni og ti prompter og scanner programmet for antallet af hold der skal fjernes. Det antal af hold der skal fjernes, bliver sat over i \textbf{\textit{number\_of\_removed\_teams}}. Denne variable gør det muligt at finde ud af hvor meget plads der skal allokeres til arrayet \textbf{\textit{removed\_teams}} i linje 13.
\cleardoublepage
{\selectlanguage{danish}
\pdfbookmark[0]{Danish title page}{label:titlepage_da}
\aautitlepage{%
  \danishprojectinfo{
    Ressourcefordeling i en idrætsforening %title
  }{%
    Resourcefordeling i en sportsklub %theme
  }{%
    Efterårssemestret 2018 %project period
  }{%
    A411 % project group
  }{%
    %list of group members
    Ane Søgaard Jørgensen\\ 
    Jakob Sønderby Kristensen\\
    Lena Said\\
    Liv Holm\\
    Michelle Volf Terpling
  }{%
    %list of supervisors
    Kurt Nørmark\\
    Hanne Bohnstedt
  }{%
    1 % number of printed copies
  }{%
    \today % date of completion
  }%
}{%department and address
  \textbf{Software og Datalogi}\\
  Aalborg Universitet\\
  \href{http://www.aau.dk}{http://www.aau.dk}
}{% the abstract
  I denne rapport er ressourcefordeling i idrætsforeninger blevet undersøgt med henblik på at finde et problem, som kan løses med et program. Gennem interviews med repræsentanter for forskellige idrætsforeninger, er der fundet udfordringer med opstillingen af stævneplaner til Kidzlige floorball. Dette er ofte en besværlig og tidskrævende opgave som vil kunne gøres hurtigere og nemmere med et passende program. Et program vil ligeledes kunne minimere risikoen for fejl i stævneplanen. Ud fra de officielle regler for Kidzliga, samt et interview med floorball-foreningen Aalborg Flyers, stillet af Floorball Danmark, er der opstillet en række krav til programmet. Ved implementeringen af disse, er der skrevet et program som tillader brugeren at danne en ny stævneplan ud fra en liste af hold med tilhørende niveau. Derefter kan brugeren redigere i denne plan ved enten at tilføje eller fjerne hold. Således sikres en vis fleksibilitet, da stævneplanen let kan ændres, hvis eksempelvis et hold melder afbud. En række yderligere funktionaliteter, der skal øge fleksibiliteten af programmet er ligeledes diskuteret. Et eksempel på disse kunne være at tilføje muligheden for at redigere i stævneplanen efter nogle af kampene er blevet spillet. 
}}
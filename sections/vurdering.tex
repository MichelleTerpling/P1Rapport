\chapter{Vurdering}\label{ch:vurdering}
% Skriv om hvordan programmet ikke helt lever op til kravene og hvordan det lever op til kravene. Hvad ville vi gerne have lavet også?
I dette kapitel vurderes det hvor godt projektets program opfylder de opstillede krav. Endvidere afprøves det hvor lang tid det tager at danne en stævneplan med de to metoder, der henholdsvis gør dette hurtigt eller kvalitetsbevidst. Til sidst udforskes det hvordan programmet kan forbedres.

\section{Opfyldning af krav}
Som en del af problemløsningen i dette projekt, blev der stillet nogle krav til programmet, der udgør løsningen. Dette program er blevet gennemgået, og i dette afsnit vil der blive set nærmere på, hvilke af de krav, der blev opstillet, programmet lever op til.

\begin{itemize}
    \item \textit{En kamp varer seks minutter, og der skal være en til to minutters pause mellem hver kamp}. Dette krav bliver opfyldt, når programmet printes. Her starter tiden med det tidspunkt, brugeren har tastet ind, og bliver talt op med otte minutter for hver runde. På denne måde starter hver runde på det rigtige tidspunkt.
    \item \textit{Alle hold, der deltager, skal have cirka seks kampe.} I programmet er der defineret en symbolsk konstant \textbf{\textit{GAMES\_PR\_TEAM}}, der er sat til seks. Når der skal findes hold til en kamp igennem funktionerne \textbf{\textit{findFirstTeam}} og \textbf{\textit{findSecondTeam}}, er en af betingelserne, for at et hold udvælges, at de har spillet under seks kampe.
    \item \textit{Alle kampe i en runde skal startes og afsluttes på samme tid}. Dette krav bliver opfyldt på samme måde som det første krav, da print funktionen kun printer tidspunktet én gang for hver runde, og derfor starter hver kamp i runden på samme tid. Det er dog i sidste ende brugerens ansvar at overholde tidsplanen. 
    \item \textit{Alle hold skal have mindst én hvilekamp mellem hver kamp}. Programmet garanterer ikke at et hold får en hvilekamp mellem hver kamp, men det opfylder den anden del af kravet. Ifølge Kidzliga reglerne, tilstræbes det at hvis holdet skal spille igen med det samme, skal holdet spille på den samme bane som i runden før. Dette overholder programmet, så derfor er dette krav delvist opfyldt.
    \item \textit{Et hold må kun spille mod et andet hold, der er på det samme niveau}. Dette krav bliver opfyldt når hold sættes sammen i kampe. Det er her et fast kriterie for udvælgelsen af det andet hold i kampen, at det har samme niveau som det første hold. 
    \item \textit{Den udviklede stævneplan skal være opstillet på en præsentabel måde}. Dette krav er subjektivt, men denne projektgruppe mener, at dette krav er blevet opfyldt. Stævneplanen er sat op så der for hver runde, kommer rundenummeret først, med tidspunktet bagefter. Planen er printet ud i kronologisk rækkefølge i en liste (se bilag \ref{ch:appHlabel}). Dette gør planen mere overskuelig, og det er nemt at finde frem til en given runde. Det er dog en udfordring at finde et specifikt hold, da de ikke er sorteret, og derfor står i tilfældig rækkefølge.
    \par
    Desuden gør layoutet det også nemmere at scanne informationer ind igen. Dette er ikke en del af kravet, men det er noget, der skal tages hensyn til, når det kommer til designet af stævneplanen.
\end{itemize}

De krav, der ikke direkte blev stillet af Floorball Danmark, er lavere prioriteret end resten af kravene. Forventningen til disse er heller ikke at de nødvendigvis kan opfyldes så konsekvent som de ovenstående, mere faste krav. 
\begin{itemize}
    \item \textit{Et hold må ikke spille mere end to kampe i træk.} Ligesom det ovenstående krav, der bygges videre på, opfyldes dette krav ikke altid. Der gives point til stævneplanerne afhængig af om de overholder kravet. Den bedste plan udvælges således, men dette garanterer ikke at kravet opfyldes, da alle mulige stævnerplaner ikke bliver fundet.
    \item \textit{Hvert hold skal spille mod et andet hold igen, så få gange som muligt, og to hold må ikke spille mod hinanden to kampe i træk.} Dette krav opfyldes ikke, da andre aspekter af programmet er prioriteret højere, og der således ikke er taget højde for dette. Dette krav anses dog for at være det mindst væsentlige og vil sandsynligvis øge programmets køretid væsentligt hvis det implementeres. 
\end{itemize}

\section{Afprøvning af programmet}\label{afsnit:test}
Et krav der blev sat op ved problemformuleringsafsnittet (se kapitel \ref{ch:problemformulering}) er, at programmet skal lave en stævneplan hurtigt. For at finde ud af om dette krav er opfyldt bliver der foretaget en tidsprøve af programmet. Denne afprøvning laves for begge metoder i programmet.
\\
Afprøvningen, der kan findes i kodeeksempel \ref{code:timeTest}, er foretaget på funktionen \textbf{\textit{generateTournament}} der kan ses i kildekode \ref{code:generateTournament} på side \pageref{code:generateTournament}. Da visse dele af denne funktion er afhængig af brugeren, og dermed dennes tidsforbrug, er de ikke blevet afprøvet. Den del af processen, der er afprøvet, er fra linje 27, og frem til og med linje 48. Den hurtigste metode afprøves først.
Det bemærkes, at hver afprøvning blev foretaget fem gange, og det længste tidsforbrug er præsenteret her. Denne del af funktionen tager 0,306 sekunder. Det er også afprøvet, hvor langt tid funktionen \textbf{\textit{printProgram}} tager om at skrive stævneplanen ud til terminalen. Denne proces tog 0,195 sekunder.
\\
% Opsamling af den første prøve
I en virkelig situation skal der selvfølgelig tages højde for, at brugeren skal bruge tid til at indtaste indformation. Men det antages, at selv med dette yderligere tidforbrug lagt til, vil programmet være hurtig nok til at udarbejde en stævneplan, lige før et stævne begynder. Kravet bliver derfor opfyldt af programmet.
\\\\
% Test af den "bedste metode"
Den anden metode bruger længere tid, men laver en bedre stævneplan. Tiden denne proces tager er også afprøvet. Processen der afprøves er den samme, som i den første prøve. Dog eksekveres en anden del af denne kode, når det vælges at bruge denne metode.  
Denne proces er også kørt fem gange, men da tidsforbruget af denne proces kan varierer, er både den længste og korteste tid, præsenteret. Denne metode bliver afprøvet med samme antal baner, samme starttidspunkt og den samme fil med holdnavne og niveau, som blev brugt i den første prøve. Disse prøver afspejler derfor ikke den tid, det kan forventes at tage for en hver stævneplan. Da tidsforbruget kan ændre sig afhængig af hvormange hold og baner der er.
Den korteste tid, der blev brugt til at lave en bedre stævneplan er 52,289 sekunder. Hvorimod den længst brugte tid er 1135,767 sekunder, svarende til ca. 19 minutter. 
\\
% Opsamling på prøverne
Prøverne viser at der er en betydelig forskel på den hastigheden af den hurtige metode, og den metode der laver den bedste stævneplan. Det fremstår også at der kan være store variationer i metoden til at lave den bedste stævneplan. Det er grundet at planen sammensættes tilfældigt, og derefter bedømmes. Hvis de ikke er tilfredsstillende bliver dele af planen igen sammensat tilfældigt indtil det er tilfredsstillede. 
\\

\begin{listing}[H]
\begin{minted}[frame=lines, framesep=3mm, baselinestretch=1, linenos, bgcolor=LightGray]{c}

clock_t begin = clock();

/* Proces der testes */

clock_t end = clock();
double time_spent = (double)(end - begin) / CLOCKS_PER_SEC;

\end{minted}
\captionof{Kodeeksempel}{Kode der bruges til at måle tidsforbruget af en proces. Dette gøres ved at måle antallet af clock-ticks, der bruges af en proces, og ved brug af \textbf{\textit{CLOCKS\_PER\_SEC}} omregnes dette til sekunder. Dette kodeeksempel bruger standardbiblioteket time.h}
\label{code:timeTest}
\end{listing}

\section{Videreudvikling}
Under udviklingen af programmet blev der lavet en liste over egenskaber, programmet kunne have for at øge fleksibiliteten, effektiviteten og brugervenligheden af programmet. Flere af disse egenskaber blev integreret i programmet, men mange blev udeladt grundet prioritering. Der var andre egenskaber, der blev vurderet til at være vigtigere at fokusere på. I dette afsnit vil de udeladte egenskaber blive gennemgået og diskuteret. 
\\\\
En af de store ting der kunne tilføjes til programmet, er backtracking, når der genereres stævneplaner. Som det er i det nuværende program, sammensættes en runde på ny, hver gang en af reglerne bliver brudt, altså når \textbf{\textit{no\_go\_count}} > 0. Når den har sammensat den samme runde \textbf{\textit{CHECK\_NUM}} gange, starter funktionen \textbf{\textit{createTournament}} helt forfra. Det betyder, at selvom algoritmen har nået den sidste runde og finder en sammensætning, der bryder en regel, fjerner den alle de andre runder, som fungerede fint, og starter forfra. \textbf{\textit{CHECK\_NUM}} er sat til et relativt stort tal, da programmet skal have en chance til at finde noget, der ikke bryder med reglerne.
\par
Dette kunne optimeres, ved at algoritmen i stedet finder det sted i stævneplanen, hvor sammensætningen forudsætter, at det går galt senere hen. Det kunne gøres, ved at algoritmen gik tilbage i stævneplanen runde for runde og sammensatte nye runder undervejs. Da algoritmen er pseudo tilfældig, vil der blive genereret en ny stævneplan, som muligvis ikke bryder reglerne. 
\par
På denne måde ville alle de runder, der fungerer ikke blive fjernet hver gang en regel brydes for mange gange, og der vil derfor ikke være så meget spildt arbejde. Dette er dog ikke blevet implementeret, da det blev prioriteret lavere end at få et program, der kunne lave en funktionel stævneplan. Hvis der også skulle have været fokus på at optimere programmet fuldstændigt, ville det have taget tid fra andre dele af projektet.
\\\\
For at øge programmets fleksibilitet, kunne der tilføjes en mulighed for at lave ændringer i planen for et igangværende stævne uden at ændre på de kampe, der allerede er spillet. Der kunne opstå situationer, hvor et hold må tage hjem før tid eller er forsinkede. I disse situationer ville det være optimalt at kunne ændre på så lidt som muligt, så de hold, der allerede har spillet, ikke får seks nye kampe.
\par 
Dette kunne implementeres ved at give hvert hold et start og et sluttidspunkt, for hvornår de kan deltage i stævnet. Funktionerne, der tager sig af at sammensætte stævneplanen, skulle så også tage højde for, at hvert hold er til stede, når runderne bliver sammensat. Yderligere, skal programmet være i stand til at reproducere den del af stævneplanen, der allerede er blevet spillet. Dette kunne enten gøres ved at gemme det seed, der blev brugt til at generere stævneplanen, eller programmet kan scanne filen med den eksisterende stævneplan, indtil det kommer til den rigtige runde.
\par
Der kan dog opstå problemer, hvis et hold først kan komme sent på dagen, da holdet kunne ende med at skulle spille alle sine kampe lige efter hinanden. Dette ville bryde med nogle af de krav, der er blev opstillet for programmet. Det er en af grunden til at dette ikke blev implementeret. Den anden grund er at dette ikke blev prioriteret lige så højt, som muligheden for at kunne tilføje og slette hold i stævneplanen. Der findes eksempler fra interviewet med Aalborg Flyers (se bilag \ref{ch:appClabel}), på at det kan være nødvendigt at tilføje og fjerne hold.
\\\\
Endnu en måde, programmets brugervenlighed kunne øges, var, hvis man kunne generere en ny stævneplan direkte fra print menuen. Som programmet er lige nu, skal man starte forfra med at taste de nødvendige informationer ind, før man kan få genereret en ny stævneplan. Dette er ikke optimalt, da brugeren kan have brug for at generere et nyt program, hvis de ikke er tilfredse. 
\par
Denne funktionalitet blev udeladt, da det blev set som et mindre problem. De fleste brugere ville antagelsesvis generere et program, og så rette det til med \textbf{\textit{editMenu}}. Igennem denne menu kan der også genereres nye stævneplaner baseret på holdene i den eksisterende stævneplan.
\\\\
Brugervenligheden og fleksibiliteten kunne også øges ved at give brugeren mulighed, for at indsætte pauser i stævneplanen. I sin nuværende version, printer programmet hele stævneplanen ud, og der indsættes ingen pauser, ud over de to minutter, der er mellem hver runde. Der kunne sagtens være situationer, hvor brugeren ønsker at indsætte længere pauser, eksempelvis frokost. Programmet skulle så også tage højde for, at der nogle steder kan stå \textit{Pause}, som det skal se bort fra. Dette problem kan godt løses, men det er igen ikke prioriteret lige så højt som andre dele af programmet, da holdene har pauser under deres hvilekampe, og derfor ikke typisk har brug for længere pauser. 
\\\\
En anden måde at hjælpe brugeren ville være at tilføje en funktion, der tjekker for hvor lang tid den stævneplan, der er ved at blive lavet, vil komme til at tage. Hvis stævnet starter om morgenen klokken 9:00 og slutter klokken 23:00, kunne der være sket en fejl fra brugerens side. Brugeren kunne have tastet det forkerte antal baner ind eller det forkerte starttidspunkt. Som programmet er lige nu, skal brugeren selv være bevidst om at finde disse fejl. Dog er programmet bygget op på den måde, at man kan få lov til at gennemse stævneplan før man gemmer den. Derfor vurderes det til at være en mindre væsentlig tilføjelse.  
\\\\
%Den stævneplan, der bliver genereret med dette program, er meget stringent i forhold til tiden. Der er sat præcis otte minutter af til hver runde, og bliver denne overskredet, falder resten af stævneplanen fra hinanden. Derfor kunne det være en god ide at tilføje en indbygget tidstager, så brugeren ikke selv skal koordinere, hvor lang tid der går mellem hver runde. 
%\par
%Dette kunne gøres ved at lave en funktion, der scanner stævneplanen igennem, og finder de tidspunkter, hvor hver runde starter. Derefter kunne en anden funktion vise, hvor lang tid der er tilbage af den nuværende runde, og så lave en lyd når kampen er ovre. Herefter kunne den lave en anden lyd, når pausen er forbi, og den næste runde skal starte. På denne måde forbliver hele stævnet forhåbentligt synkroniseret, og der er ikke nogen, som går i gang for sent eller for tidligt.
%\par
%Denne funktionalitet er dog ikke blevet implementeret, da det blev set som en ekstra ting, programmet ikke har brug for at gøre for at kunne fungere.

\subsection*{Opsamling} 
Programmet, der er blevet lavet, lever ikke op til alle kravene, men størstedelen af kravene er der taget højde for. Programmet løser dog stadigvæk problemformuleringen, i forhold til at det kan lave og redigere en stævneplan. Redigeringen er dog begrænset, da der kun kan tilføjes og fjernes hold. I kapitlet er det også beskrevet, hvordan man kan forbedre programmet, og gøre det mere fleksibelt.